% ============================
% CONFIGURACIÓN DEL DOCUMENTO
% ============================
\documentclass[spanish]{thesis} % Clase de documento específica para tesis

% ============================
% PAQUETES UTILIZADOS
% ============================
   % Plantilla específica para tesis
\usepackage{titlesec}        % Personalización de títulos y secciones
\usepackage{pgfplots}        % Gráficos en coordenadas matemáticas
\usepackage{graphicx}        % Inclusión de imágenes
\usepackage{appendix}        % Manejo de anexos
\usepackage{rotating}        % Rotación de elementos
\usepackage{tabularray}      % Tablas avanzadas
\usepackage{lipsum}          % Texto de relleno
\usepackage{tablefootnote}   % Notas al pie en tablas
\usepackage{tabularx}
\usepackage{array}
\usepackage{enumitem}
\usepackage{tcolorbox}
\usepackage{booktabs} 
% Paquetes matemáticos
\usepackage{amssymb}
\usepackage{latexsym}
\usepackage{amsmath}
\usepackage{csquotes}
\usepackage{newtxtext, newtxmath}
\usepackage[hang,flushmargin]{footmisc}
% Paquetes para algoritmos
\usepackage{algorithm}
\usepackage{algorithmic}

% Configuración de código fuente con listings
\usepackage{listings} % Resaltado de código


\lstset{
	extendedchars=true,
	language=Python,
	basicstyle=\footnotesize\ttfamily,
	showstringspaces=false,
	showspaces=false,
	numbers=left,
	numberstyle=\footnotesize,
	numbersep=9pt,
	tabsize=2,
	breaklines=true,
	showtabs=false,
	captionpos=b,
	basicstyle=\ttfamily, 
	keywordstyle=\bfseries,
	morekeywords={self,import,print},
	xleftmargin=15pt,
	xrightmargin=0pt,
	emph={MyClass,__init__}
}

% ============================
% DEFINICIÓN DE COMANDOS
% ============================
\newcommand\blankpage{%
	\null
	\thispagestyle{empty}%
	\addtocounter{page}{-1}%
	\newpage}

% Configuración de gráficos con PGFPLOTS
\pgfplotsset{
	textnumber/.style={
		/pgf/number format/.cd,
		fixed,
		fixed zerofill,
		precision=4,
		1000 sep={ },
	},
}

% Definición de operadores matemáticos personalizados
\newtheorem{definition}{Definición}
\DeclareMathOperator*{\argmax}{arg\,max}
\DeclareMathOperator*{\argmin}{arg\,min}

% Anular todos los comandos obligatorios
\makeatletter
\def\@keywords{}
\def\@acknowledgment{}
\def\@abstract{}
\makeatother

% Formato de subpárrafos
\titleformat{\subparagraph}
{\normalfont\normalsize\bfseries}{\thesubparagraph}{1em}{}
\titlespacing*{\subparagraph}{\parindent}{3.25ex plus 1ex minus .2ex}{.75ex plus .1ex}

% ============================
% INFORMACIÓN DEL DOCUMENTO
% ============================
\addauthor{Daniel Rojas Grass}
\title{{\normalsize Multiagente Conversacional para la interacción con los datos del transporte marítimo recogidos en el Diario de la Marina.}}
\ucicenter{Vicerrectoría de Investigación y Postgrado}
\facultynum{1}
\addtutor[Tutor]{Dr.C. Orlando Grabiel Toledano López}
\addtutor[Tutores]{MsC. Olga Yarisbel Rojas Grass}


% ============================
% ESTRUCTURA DEL DOCUMENTO
% ============================
\begin{document}
	
	% ========================
	% PORTADA Y SECCIONES PRELIMINARES
	% ========================
	\begin{titlepage}
		\begin{figure}
			\centering
			\includegraphics[width=0.25\linewidth]{images/uci}
		\end{figure}
		\centering{\textsc{\large UNIVERSIDAD DE LAS CIENCIAS INFORMÁTICAS}     \\
			\textsc{\large VICERRECTORÍA DE INVESTIGACIÓN Y POSTGRADO} \\
			\textsc{\large FACULTAD DE TECNOLOGIAS LIBRES} \\}
		\vspace{3cm}
		\centering\textsc{\LARGE Multiagente Conversacional para la interacción con los datos del transporte marítimo recogidos en el Diario de la Marina.\\}
		\vspace{2cm}
		\centering{\bfseries Trabajo de diploma para optar por el título de \\
			Ingeniero en Ciencias Informáticas\\}
		\vspace{2.5cm}
		\centering{\bfseries Autor:\\}
		\centering{Daniel Rojas Grass\\}
		\vspace{0.5cm}
		\centering{\bfseries Tutores:\\}
		\centering{
			Dr.C. Orlando Grabiel Toledano López\\
			MsC. Olga Yarisbel Rojas Grass\\
		}	
		\vspace{1.5cm}
		\centering{ \bfseries La Habana,2025}
	\end{titlepage}
	
	\pagenumbering{roman}
	\authorshipdeclared

\parskip 10pt 
\spacing{1.5} 
\setlength{\parindent}{0pc}

El autor del trabajo de diploma con título  \textit{\textbf{“Multiagente conversacional para la interacción con los datos del transporte marítimo recogidos en el Diario de la Marina”}}, concede a la Universidad de las Ciencias Informáticas los derechos patrimoniales de la investigación, con carácter exclusivo. De forma similar se declara como único autor de su contenido.

Y para que así conste, firmo la presente declaración jurada de autoría en La Habana a los \rule{8mm}{0.2mm} días del mes de \rule{30mm}{0.2mm} del año \rule{15mm}{0.2mm}.


\vspace{1in}

\begin{center}
	\rule{60mm}{0.3mm} \\
	\textbf{Daniel Rojas Grass}
\end{center}

%\bigskip
\vspace{0.5in}

\begin{center}
	\begin{tabular}{cp{0.5in}c}		
		\rule{65mm}{0.3mm}      &   & \rule{65mm}{0.3mm}        \\
		%\cargoTutor &   & \cargoCoTutor \\
		\textbf{Dr.C. Orlando Grabiel Toledano López}      &   & \textbf{MsC. Olga Yarisbel Rojas Grass}
	\end{tabular}
\end{center}

%\bigskip
\vspace{0.5in}

	\contactdata

\parskip 10pt \spacing{1.5} \setlength{\parindent}{0pc}

Curriculum e información de contacto del tutor: nombre y apellidos, títulos académicos, formación de postgrado recibida, lugar de trabajo, responsabilidades laborales asumidas, experiencia profesional, líneas de trabajo y/o investigación, correo electrónico, perfiles en redes profesionales

Curriculum e información de contacto del asesor: nombre y apellidos, títulos académicos, formación de postgrado recibida, lugar de trabajo, responsabilidades laborales asumidas, experiencia profesional, líneas de trabajo y/o investigación, correo electrónico, perfiles en redes profesionales

Curriculum e información de contacto del consultante: nombre y apellidos, títulos académicos, formación de postgrado recibida, lugar de trabajo, responsabilidades laborales asumidas, experiencia profesional, líneas de trabajo y/o investigación, correo electrónico, perfiles en redes profesionales
	\agradecimient

\parskip 10pt \spacing{1.5} \setlength{\parindent}{0pc}

A lo largo del desarrollo de esta tesis, he contado con el apoyo, guía y aliento de muchas personas a quienes deseo expresar mi más sincero agradecimiento.

En primer lugar, agradezco profundamente a mis tutores Dr.C. Orlando Grabiel Toledano López y MsC. Olga Yarisbel Rojas Grass, por su constante acompañamiento, paciencia, orientación y valiosos consejos que enriquecieron cada etapa de este trabajo y que sin su apoyo este proyecto no se hubiera llevado a cabo.

A mis profesores/as de la Facultad de Técnologias Libres, por compartir sus conocimientos y por fomentar en mí el pensamiento crítico, la disciplina académica y el amor por la investigación.

A Jesús Enrique Fernández Prieto por sus valiosos consejos de Ingienería de Software.

	\dedication

\parskip 10pt \spacing{1.5} \setlength{\parindent}{0pc}

A mis padres, por una crianza basada en el respeto, la libertad y una confianza incondicional en mi vocación y mis sueños. Gracias por enseñarme que se puede avanzar con firmeza cuando se tiene constancia y disciplina. Gracias por dejarme ser el dueño de mis decisiones y contar siempre con su apoyo.

A mi hermana Yari, por brindarme los recursos, el apoyo y las condiciones necesarias para crecer. Gracias por creer en mí, y por ser una guía silenciosa pero constante en mi camino.

A Sami, por ser parte esencial de este logro. Por tu compañía en los momentos buenos, en mis noches programando y en los momentos difíciles de frustración, por tu ternura, tu fe en mí, y por recordarme cada día que no estoy solo. Esta tesis también es uno de tus logros.

A mi familia y amigos, por estar presentes a su manera, por las palabras de aliento, los gestos sinceros y el cariño, esto siempre ha sido el impulso para ser mejor cada día.

A todos ustedes, les dedico este esfuerzo convertido en realidad.
	\resumen

\parskip 10pt \spacing{1.5} \setlength{\parindent}{0pc}

En la era digital, vastos archivos históricos como el \textit{Diario de la Marina} (1844-1960), crónicas del transporte marítimo cubano, permanecen a menudo como tesoros de información subutilizados debido a la complejidad de su acceso y análisis. Esta investigación se propuso desbloquear este potencial mediante el desarrollo de un innovador sistema multiagente conversacional. El objetivo fue transformar los datos digitalizados, frecuentemente imperfectos por errores de OCR y lenguaje antiguo, en conocimiento interactivo y contextualizado para historiadores y académicos. Empleando la metodología ágil \textit{Extreme Programming} (XP), se construyó una arquitectura de microservicios robusta (React, Django REST \textit{Framework}, \textit{FastAPI}) que orquesta un sistema de IA especializado. Este núcleo inteligente, basado en la arquitectura de Recuperación Aumentada por Generación (RAG) y potenciado por \textit{LangChain}, \textit{embeddings} y bases de datos vectoriales FAISS, permite a los usuarios dialogar en lenguaje natural con la historia, obteniendo no solo respuestas textuales precisas sino también visualizaciones gráficas dinámicas. Las pruebas exhaustivas confirmaron la funcionalidad completa de todos los requisitos y la seguridad. Si bien se cumplieron los objetivos de rendimiento, la latencia inherente a la IA síncrona se identificó como un desafío a optimizar, concluyendo que la sinergia entre sistemas multiagente e IA generativa ofrece una vía prometedora y viable para revitalizar el patrimonio documental.

\textbf{Palabras clave:} Datos Históricos, Inteligencia Artificial Conversacional, Procesamiento de Lenguaje Natural, Recuperación Aumentada por Generación, Sistema Multiagente.
	\englishabstract

\parskip 10pt \spacing{1.5} \setlength{\parindent}{0pc}

Text

\textbf{Keywords:} three to five word
	
	% Agregar el índice aquí
	\cleardoublepage
	\phantomsection
	\addcontentsline{toc}{chapter}{Índice general}
	\tableofcontents
	\listoffigures
	\listoftables
	\cleardoublepage
	
	
	% ============================
	% CAPÍTULOS DE LA TESIS
	% ============================
	\pagenumbering{arabic}
	\introduction

\parskip 10pt \spacing{1.5} \setlength{\parindent}{0pc}

La digitalización masiva de documentos históricos ha transformado el acceso a fuentes de información que, durante siglos, permanecieron confinadas a archivos físicos. Este proceso ha dado lugar a una explosión de datos digitales provenientes de colecciones tan diversas como periódicos históricos, manuscritos y registros oficiales, incluyendo ejemplos emblemáticos como el Diario de la Marina (Cuba, 1844–1960), un testimonio clave de la historia política y cultural del Caribe. Sin embargo, esta transición del medio impreso al digital no ha estado exenta de desafíos. La conversión de estos materiales ha generado grandes volúmenes de datos no estructurados, caracterizados por su heterogeneidad, falta de metadatos estandarizados y dificultades para su procesamiento automatizado. Este fenómeno plantea problemas tanto técnicos —como la preservación a largo plazo y la interoperabilidad de formatos— como conceptuales —como la extracción de conocimiento significativo y su contextualización para fines académicos y culturales—.
En un mundo donde la cantidad de información digital crece exponencialmente, la incapacidad de estructurar y analizar estos datos de manera eficiente limita su potencial como recurso para la investigación histórica, el análisis lingüístico y la comprensión de patrones sociales del pasado. Estudios recientes subrayan la necesidad de desarrollar enfoques innovadores que no solo preserven estos acervos digitales, sino que también los transformen en repositorios accesibles y funcionalmente útiles. No obstante, las técnicas tradicionales de procesamiento de datos, basadas en métodos manuales o semi-automatizados, resultan insuficientes frente a la escala y complejidad de estas colecciones. La falta de estructura inherente a los documentos digitalizados —a menudo escaneados como imágenes o textos sin formato— obstaculiza la aplicación de herramientas analíticas avanzadas, dejando gran parte de este conocimiento histórico en un estado de latencia digital.
En este contexto, la inteligencia artificial emerge como una solución prometedora para superar estas limitaciones. En particular, los modelos de lenguaje de gran escala (LLMs), gracias a su capacidad para interpretar y generar texto con un alto grado de sofisticación, ofrecen una oportunidad única para abordar la problemática de los datos históricos no estructurados. Sin embargo, su implementación aislada no basta para resolver la diversidad de tareas involucradas, como la transcripción, categorización, enriquecimiento semántico y búsqueda contextual. Es aquí donde un enfoque multiagente, impulsado por la colaboración de múltiples LLMs especializados, puede marcar una diferencia significativa. Este tipo de sistema tiene el potencial de coordinar esfuerzos entre agentes diseñados para tareas específicas, optimizando así la transformación de datos brutos en conocimiento estructurado y accesible.\\
El Diario de la Marina, publicado en La Habana entre 1844 y 1960, se erigió como un referente clave de la prensa cubana, autodenominado «El decano de la prensa cubana» tras suceder a El Noticioso y Lucero de La Habana. Este periódico de carácter conservador no solo documentó eventos políticos y sociales de su tiempo, sino que también registró información valiosa sobre actividades económicas, como el transporte marítimo, un pilar fundamental de la historia comercial y cultural de Cuba. La digitalización de sus páginas, impulsada por la necesidad de preservar este patrimonio histórico, ha permitido rescatar y almacenar de manera segura contenidos que, de otro modo, podrían haberse perdido debido al deterioro físico de los ejemplares originales. Técnicas como el Reconocimiento Óptico de Caracteres (OCR) han facilitado la conversión de imágenes escaneadas en datos digitales editables y buscables, mejorando la accesibilidad a esta riqueza informativa. Sin embargo, este proceso presenta limitaciones significativas. Factores como la diversidad de tipografías antiguas, la calidad variable de impresión y el estado de conservación de los documentos generan datos no estructurados, con errores de transcripción y sin una organización clara. En el caso específico de los datos relacionados con el transporte marítimo —como rutas, cargamentos, puertos y fechas—, esta falta de estructura dificulta su análisis sistemático y su uso en investigaciones históricas o económicas. Por ende, la mera digitalización no basta: se requieren procesos avanzados de interpretación y contextualización para transformar estos datos en conocimiento útil, especialmente cuando se busca responder a consultas específicas de los usuarios.

\textbf{Planteamiento del problema científico}

A partir de la situación problemática descrita se identifica el siguiente \textbf{problema científico}:

\textit{¿Cómo interpretar y contextualizar los datos no estructurados del transporte marítimo recogidos en el Diario de la Marina, a partir de consultas del usuario expresadas en lenguaje natural?}

Este desafío implica superar las barreras de la imprecisión del OCR, la ausencia de metadatos estandarizados y la complejidad de extraer significado de textos históricos, todo ello mientras se habilita una interacción intuitiva y efectiva con los usuarios mediante un sistema conversacional.

Con lo anterior, se define como \textbf{objeto de estudio} la visualización de la transformación de datos aplicando técnicas de inteligencia artificial. Como \textbf{campo de acción} los sistemas multiagentes para la transformación de datos no estructurados en conocimiento estructurado.

\textbf{Objetivo general de la investigación}

\textit{Desarrollar un sistema multiagente conversacional que permita interpretar y contextualizar automáticamente los datos no estructurados del transporte marítimo recogidos en el Diario de la Marina, facilitando respuestas precisas y relevantes a consultas expresadas en lenguaje natural por los usuarios.}

\textbf{Objetivos específicos}

A partir del planteamiento del objetivo de investigación, se definen los siguientes objetivos específicos:
	\begin{itemize}
		\item Identificar los referentes teóricos y metodológicos sobre la transformación de información no estructurada a lenguaje natural, revisando enfoques de inteligencia artificial y sistemas conversacionales aplicados a datos históricos.
		\item Realizar la identificación de requisitos, análisis y diseño del sistema multiagente conversacional, definiendo sus componentes, interacciones y flujos de trabajo.
		\item Implementar un sistema multiagente conversacional que contribuya a la transformación y el análisis de la información no estructurada extraída del Diario de la Marina, enfocándose en los datos del transporte marítimo.
		\item Validar el correcto funcionamiento del sistema multiagente conversacional, aplicando métricas de evaluación y pruebas de software para garantizar su calidad, precisión y usabilidad.
	\end{itemize}
	
\textbf{Hipótesis científica}

La hipotesis va aqui .....

\textbf{Métodos de Investigación}
	
	Para llevar a cabo esta investigación se aplicaron métodos teóricos y empíricos de la investigación científica, los cuales se relacionan con el desarrollo de un sistema multiagente conversacional para la transformación e interacción con datos históricos. A continuación, se detallan:
	
\textbf{Métodos Teóricos}
	
	\begin{enumerate}
		\item \textbf{Analítico-sintético.} Permitió analizar, sintetizar y evaluar el proceso de transformación de datos no estructurados provenientes del \textit{Diario de la Marina}, desde un enfoque centrado en la aplicación de técnicas de procesamiento de lenguaje natural y sistemas multiagentes. Con este método se identificó la esencia del problema de investigación, analizando los componentes del proceso de digitalización, las limitaciones de las técnicas actuales de reconocimiento óptico de caracteres (OCR) y las necesidades de contextualización de los datos del transporte marítimo para su uso efectivo.
		
		\item \textbf{Hipotético-deductivo.} Se empleó para identificar las variables clave involucradas en la interacción conversacional con datos históricos y sus interrelaciones, especialmente aquellas relacionadas con el diseño de sistemas multiagentes y el procesamiento de consultas en lenguaje natural. Este método facilitó la formulación de supuestos sobre cómo los agentes colaborativos pueden mejorar la interpretación y estructuración de la información no estructurada.
		
		\item \textbf{Histórico-lógico.} Se aplicó para revisar la evolución de las tecnologías de digitalización de documentos históricos, el desarrollo de sistemas conversacionales basados en inteligencia artificial y el uso de datos del transporte marítimo en contextos históricos. Este enfoque permitió reconocer los avances teórico-prácticos en el área, así como las limitaciones actuales en la gestión de datos no estructurados extraídos de fuentes como el \textit{Diario de la Marina}.
	\end{enumerate}
	
	\subsection*{Métodos Empíricos}
	
	\begin{enumerate}
		\item \textbf{Análisis documental.} Consistió en la revisión de la literatura relacionada con la transformación de datos no estructurados, el diseño de sistemas multiagentes y las aplicaciones de modelos de lenguaje en la interpretación de textos históricos. Incluyó el estudio de enfoques, algoritmos y herramientas de inteligencia artificial utilizadas en tareas de procesamiento de lenguaje natural y extracción de conocimiento a partir de documentos digitalizados.
		
		\item \textbf{Experimental.} Se empleó para comprobar los resultados del sistema multiagente conversacional desarrollado, evaluando su capacidad para interpretar consultas en lenguaje natural y contextualizar los datos del transporte marítimo del \textit{Diario de la Marina}. Se compararon los resultados obtenidos mediante métricas estándares de evaluación (como precisión, recall y F1-score), ajustadas a la calidad de las respuestas generadas y la satisfacción del usuario.
	\end{enumerate}
	
	
\textbf{Estructura del documento}

El documento está organizado en introducción, tres capítulos, conclusiones, recomendaciones, bibliografía y anexos. A continuación se describe el contenido abordado en cada capítulo:

\begin{itemize}
	\item \textbf{Capítulo 1.Fundamentos Teóricos y Contextuales de la Transformación de Datos Históricos y Sistemas Multiagentes Conversacionales}:
	Se realiza un estudio y análisis de los diferentes métodos y técnicas para el procesamiento de datos no estructurados provenientes de documentos históricos, con énfasis en la digitalización del Diario de la Marina. Se analizan los fundamentos teóricos relacionados con la transformación de datos mediante técnicas de inteligencia artificial, centrándose en el uso de sistemas multiagentes y modelos de lenguaje para interpretar textos históricos. De igual manera, se revisan los principales referentes teóricos sobre el procesamiento de lenguaje natural (PLN), la estructuración de información no estructurada y la interacción conversacional con usuarios. Finalmente, se evalúan los desafíos específicos asociados a los datos del transporte marítimo (rutas, puertos, fechas) extraídos de fuentes digitalizadas, identificando las limitaciones de las técnicas actuales como el Reconocimiento Óptico de Caracteres (OCR).
	\item \textbf{Capítulo 2.Diseño e Implementación del Sistema Multiagente Conversacional}:
	En este capítulo, se desarrolla un sistema multiagente conversacional diseñado para interpretar y contextualizar automáticamente los datos no estructurados del transporte marítimo del Diario de la Marina. Se modelan y describen las arquitecturas de los agentes involucrados, incluyendo un agente de preprocesamiento (corrección de errores de OCR), un agente de contextualización (enriquecimiento semántico) y un agente conversacional (interacción con el usuario en lenguaje natural). Se analiza el costo computacional del sistema desarrollado y sus variantes, determinadas por las tecnologías y algoritmos empleados en cada fase (como modelos de lenguaje preentrenados y frameworks de multiagentes). Finalmente, se presentan las conclusiones parciales del capítulo, destacando las decisiones de diseño y los resultados preliminares de la implementación.
	\item \textbf{Capítulo 3.Validación y Análisis de Resultados}:
	Se desarrolla un conjunto de experimentos utilizando una muestra representativa de datos digitalizados del Diario de la Marina, enfocándose en registros del transporte marítimo, y se discuten los principales resultados en cuanto a la eficacia del sistema multiagente al procesar consultas en lenguaje natural. Con la arquitectura óptima seleccionada, se evalúa el sistema propuesto en escenarios prácticos, como la respuesta a preguntas sobre rutas marítimas, puertos y fechas históricas. Primero, se valida el sistema con una muestra controlada de datos extraídos del periódico y se comparan los resultados con enfoques tradicionales del estado del arte (por ejemplo, búsquedas manuales o sistemas no conversacionales). Se realiza un análisis exploratorio de los datos históricos, incluyendo preprocesamiento, corrección de errores y extracción de características relevantes para estructurar la información. Finalmente, se presentan los resultados experimentales, evaluando métricas como precisión, recall y satisfacción del usuario, y se discuten las implicaciones para la investigación histórica y la gestión de patrimonios digitales.
\end{itemize}


	











 % Introducción
	\chapter{Fundamentos Teóricos y Contextuales de la Transformación de Datos Históricos y Sistemas Multiagentes Conversacionales}
\label{chap:chapter1}

 % Capítulo 1
	\chapter{Análisis, diseño e implementación de la propuesta de solución}
\label{chap:chapter2}

%\section*{Introducción}
%\addcontentsline{toc}{section}{Introducción}
Este capítulo presenta la propuesta de solución para abordar el problema identificado en la investigación: la interpretación y contextualización de datos no estructurados del transporte marítimo extraídos del Diario de la Marina (1844–1960). Aquí se detallan los requisitos funcionales (RF) y no funcionales (RNF) que el sistema debe cumplir, así como una descripción exhaustiva de la solución propuesta, incluyendo las historias de usuario y las tarjetas CRC (Clase-Responsabilidad-Colaboración). Además, se analizan los patrones arquitectónicos y de diseño seleccionados para estructurar la aplicación, junto con las buenas prácticas de codificación asociadas a las tecnologías empleadas. Finalmente, se incluyen el diagrama de clases y el diagrama de componentes, que ilustran los elementos clave que integran esta propuesta. Este capítulo sienta las bases técnicas y conceptuales para la implementación y validación descritas en capítulos posteriores, alineándose con los objetivos específicos de diseño e implementación establecidos en la introducción.

\section{Descripción de la Propuesta de Solución}
\label{sec:propuesta_solucion}

La solución propuesta aborda la interacción con los datos históricos del transporte marítimo del \textit{Diario de la Marina} (1844-1960) mediante el desarrollo de una aplicación web moderna, separando claramente las responsabilidades entre la interfaz de usuario, la lógica de negocio y el procesamiento avanzado de lenguaje natural. Esta arquitectura se compone de tres elementos principales: un \textit{frontend} desarrollado con React js, un \textit{backend} API construido con Django REST Framework(DRF), y un microservicio dedicado que alberga el sistema multiagente conversacional basado en inteligencia artificial y que se expone con una api creada con FastApi. 
%Esta elección arquitectónica promueve la escalabilidad, la mantenibilidad y la separación de intereses, permitiendo que cada componente evolucione de forma independiente.

% Incluye aquí tu nueva figura de arquitectura web
\begin{figure}[htbp] % h: here, t: top, b: bottom, p: page of floats - ajusta según necesidad
	\centering
	% Asegúrate de que la ruta 'images/arquitectura_web.png' sea correcta
	\includegraphics[width=0.9\textwidth]{images/micro.png} 
	\caption{Estructura de la propuesta de solución (Fuente: elaboración propia).}
	\label{fig:arquitectura_web}
\end{figure}

La Figura \ref{fig:arquitectura_web} ilustra la arquitectura general. El flujo de interacción del usuario es el siguiente:

\begin{enumerate}
	\item El usuario interactúa con la interfaz web desarrollada en React, donde ingresa sus consultas en lenguaje natural (e.g., “¿Qué barcos llegaron de Europa en enero de 1850?”).
	\item El \textit{frontend} React envía la consulta del usuario, típicamente como una petición HTTPS, al \textit{endpoint} correspondiente del \textit{backend} de DRF.
	\item El \textit{backend} de DRF actúa como orquestador principal de la lógica de negocio. Puede gestionar autenticación de usuarios (si aplica), almacenar historial de conversaciones, y, crucialmente, procesa la solicitud entrante. Determina que la consulta requiere procesamiento por IA y la reenvía al microservicio del sistema multiagente a través de una llamada API.
	\item El microservicio del sistema multiagente recibe la consulta y ejecuta su lógica interna basada en agentes para procesar el lenguaje natural, buscar en la base de datos vectorial del \textit{Diario de la Marina}, contextualizar la información y generar la respuesta (texto y/o imágenes).
	\item El microservicio del sistema multiagente devuelve el resultado procesado al \textit{backend} de DRF.
	\item El \textit{backend} de DRF recibe la respuesta del microservicio, la formatea si es necesario (e.g., preparándola para la visualización), y la envía de vuelta al frontend React.
	\item Finalmente, el \textit{frontend} React recibe la respuesta del \textit{backend} de DRF y la presenta al usuario de forma clara y ordenada en la interfaz de chat, mostrando el texto y/o las imágenes generadas.
\end{enumerate}

Dentro del microservicio sistema multiagente opera la lógica conversacional avanzada, cuyo flujo interno se detalla a continuación y se ilustra en la Figura \ref{fig:flujo_mas_interno}. Este microservicio está diseñado específicamente para transformar los datos históricos no estructurados del \textit{Diario de la Marina} en conocimiento estructurado y accesible, superando las limitaciones de herramientas previas mediante la integración de LLMs\footnote{LLM (Large Language Model): Modelo de lenguaje de gran tamaño entrenado con enormes volúmenes de datos textuales, capaz de comprender y generar lenguaje natural con un alto nivel de coherencia, usado en tareas como la traducción automática, el resumen de textos o la generación de respuestas conversacionales.}, RAG\footnote{RAG (Retrieval-Augmented Generation): Técnica que combina modelos generativos con mecanismos de recuperación de información, permitiendo al modelo acceder a fuentes externas relevantes durante la generación de respuestas, mejorando así la precisión y actualidad de los resultados producidos.} y una coordinación eficiente entre agentes especializados.

\begin{figure}[htbp] 
	\centering
	\includegraphics[width=1\textwidth]{images/mas-final.png} 
	\caption{Flujo Interno del Microservicio Multiagente Conversacional (Fuente: elaboración propia).}
	\label{fig:flujo_mas_interno}
\end{figure}

El proceso interno del microservicio (Figura \ref{fig:flujo_mas_interno}) inicia cuando recibe una consulta del \textit{backend} DRF. Esta consulta es manejada por el Agente Moderador, que actúa como coordinador central dentro del microservicio. El Agente Moderador extrae palabras clave (e.g., “barcos”, “Europa”, “enero 1850”) y determina la intención del usuario. En la mayoría de los casos, delega la tarea de búsqueda al Agente PandasAi, que realiza la extracción de datos estructurados (como fechas, nombres de capitanes, puertos) desde el archivo CSV que contiene los datos procesados del \textit{Diario de la Marina}. Solo en casos específicos, cuando la consulta requiere información detallada sobre el contenido de la carga de los barcos, el Agente Moderador coordina con el Agente Recuperador, que convierte las palabras clave en \textit{embeddings} y consulta la base de datos vectorial especializada para recuperar fragmentos relevantes. Los datos recuperados, ya sea por el Agente PandasAi o el Agente Recuperador, se devuelven al Agente Moderador.
Posteriormente, el Agente Moderador envía la información recuperada, junto con la intención del usuario, al Agente Contextualizador. Este agente evalúa si se necesita una respuesta textual, una visualización gráfica (e.g., un gráfico de la frecuencia de llegada de barcos por mes), o ambas. Si se requieren gráficos, el Agente Contextualizador genera instrucciones que se envían al Agente PandasAi. Este agente crea un script Python para generar la visualización, valida su corrección y genera una imagen (e.g., PNG). La imagen resultante se retorna al Agente Contextualizador. Si no se requieren gráficos, el Agente Contextualizador enriquece la información textual. Antes de enviar la respuesta final al \textit{backend} DRF, el Agente Validacion revisa la coherencia y precisión del contenido generado (texto y/o imagen), comparándolo con la consulta original. Finalmente, el Agente Moderador ensambla la respuesta validada y la retorna como salida del microservicio.
Esta arquitectura desacoplada, combinando una interfaz web moderna, un backend robusto y un microservicio especializado en IA, no solo facilita la interacción del usuario y la gestión de datos, sino que también optimiza la investigación histórica, contribuye a la preservación del patrimonio documental y permite escalar los componentes de IA de forma independiente.


\section{Análisis de requisitos}

El análisis de requisitos da como resultado la especificación de las características operativas del software. Indica la interfaz de este y otros elementos del sistema, y establece las restricciones que limitan al software \cite{pressman2010practitioner}.Es una fase crucial en el proceso de desarrollo de software. Se trata de una etapa inicial en la cual un analista busca entender las necesidades del cliente y traducirlas en un conjunto de requisitos claros y bien definidos \cite{palli2023analisis}.

\subsection{Técnicas de captura de requisitos}

La definición de los requisitos del sistema se fundamenta en un proceso sistemático de captura, basado en dos técnicas ampliamente reconocidas en ingeniería de software: la entrevista y la observación, aplicadas en el contexto de los ejemplos analizados en el estado del arte (Capítulo 1). Estas técnicas, permitieron identificar las expectativas de los usuarios potenciales y las limitaciones de las soluciones existentes, asegurando que los requisitos reflejen tanto las demandas prácticas como las carencias técnicas observadas~\cite{sommerville2011software}.

\textbf{Entrevista:} Se realizaron entrevistas semiestructuradas con historiadores y académicos especializados en historia del Caribe, quienes serían usuarios finales del sistema. Las preguntas se diseñaron para explorar sus necesidades al interactuar con documentos históricos digitalizados, como el \textit{Diario de la Marina}. Por ejemplo, se les consultó: ``¿Qué tipo de información busca con mayor frecuencia en archivos históricos?'' y ``¿Qué dificultades encuentra al analizar datos marítimos de textos antiguos?''. Las respuestas destacaron la importancia de obtener respuestas contextualizadas (e.g., vinculadas a eventos históricos), la necesidad de visualizaciones gráficas para patrones comerciales y la frustración con errores de transcripción que dificultan el análisis. Estas aportaciones guiaron la definición de requisitos como la corrección de datos y la generación de gráficos.

\textbf{Análisis de Sistema Homólogos:} Mediante el análisis de sistemas existentes es
posible estudiar aplicaciones similares a la que se necesita obtener. Cuando se tiene
la concepción del funcionamiento de un software similar en cuanto a funcionalidades
y características es más sencillo identificar los requisitos del sistema que se necesita
implementar. Durante la investigación se realizó un estudio de aplicaciones similares
a la solución a desarrollar, en las cuales se observaron los diseños de sus interfaces,
las funcionalidades que ofrecen, el grado de dificultad a la hora de interactuar con
la aplicación, entre otros rasgos importantes que contribuyeran a obtener un
producto con la mejor calidad posible~\cite{sommerville2011software}. Como fuente importante
para la obtención de requisitos principales del sistema a desarrollar, se encuentra el
análisis a fondo de los sistemas homólogos estudiados y antes descritos en el
epígrafe 1.2.\\
La combinación de entrevistas y observación permitió comprender las necesidades de los usuarios con las deficiencias técnicas de las soluciones actuales, proporcionando una base sólida para los requisitos funcionales y no funcionales que se detallan a continuación.



\subsection{Requisitos funcionales}

Son declaraciones de los servicios que debe proporcionar el sistema, de la manera en que
éste debe reaccionar a entradas particulares y de cómo se debe comportar en situaciones
particulares. En algunos casos, los requerimientos funcionales de los sistemas también
pueden declarar explícitamente lo que el sistema no debe hacer~\cite{sommerville2011software}.

A continuación, se describen los requisitos funcionales específicos para el sistema multiagente conversacional propuesto:

\begin{longtable}{@{}l >{\raggedright\arraybackslash}p{4.5cm} >{\raggedright\arraybackslash}p{6.5cm} l l@{}} % Ajustados anchos
	\caption{Tabla de Requisitos Funcionales (RF)} \label{tab:requisitos_funcionales_final} \\ 
	\toprule
	\textbf{ID} & \textbf{Nombre} & \textbf{Descripción} & \textbf{Complejidad} & \textbf{Prioridad} \\ 
	\midrule
	\endfirsthead 
	
	\caption[]{Tabla de Requisitos Funcionales (RF) (Continuación)} \\ 
	\toprule
	\textbf{ID} & \textbf{Nombre} & \textbf{Descripción} & \textbf{Complejidad} & \textbf{Prioridad} \\ 
	\midrule
	\endhead 
	
	\bottomrule
	\multicolumn{5}{r@{}}{\textit{Continúa en la siguiente página...}} \\ 
	\endfoot 
	
	\bottomrule
	\endlastfoot 
	
	% --- Gestión de Usuarios y Autenticación ---
	\textbf{RF 1} & Registrar usuario & Permitir a un visitante registrarse proporcionando datos válidos (e.g.,nombre de usuario, email, contraseña), que serán almacenados de forma segura por el backend. & Media & Alta \\ 
	\textbf{RF 2} & Autenticar usuario & Permitir a un usuario registrado iniciar sesión proporcionando credenciales válidas, verificadas por el backend, generando una sesión activa. & Media & Alta \\ 
	\textbf{RF 3} & Cerrar sesión & Permitir al usuario autenticado finalizar su sesión activa en el sistema. & Baja & Media \\ 
	% --- Gestión de Conversaciones ---
	\textbf{RF 4} & Iniciar conversación & Permitir al usuario autenticado crear una nueva sesión de chat independiente de las anteriores. & Baja & Alta \\ % NUEVO
	\textbf{RF 5} & Listar conversación & Mostrar al usuario autenticado una lista de sus conversaciones previas para poder seleccionarlas y revisarlas. & Media & Media \\ % NUEVO
	\textbf{RF 6} & Visualizar conversación & Permitir al usuario seleccionar una conversación de su historial para visualizarla y continuarla. & Media & Media \\ % NUEVO
	\textbf{RF 7} & Eliminar conversación & Permitir al usuario eliminar permanentemente una conversación específica de su historial. & Media & Baja \\ 
	% --- Interfaz y Flujo del Chat ---
	\textbf{RF 8} & Insertar consulta & Permitir al usuario escribir y enviar una consulta en lenguaje natural dentro de la conversación activa. & Baja & Alta \\ 
	\textbf{RF 9} & Visualizar mensajes & Mostrar de forma clara y ordenada el diálogo (consultas del usuario y respuestas del sistema) dentro de la conversación activa. & Media & Alta \\  
	% --- Procesamiento Interno del Microservicio MAS ---
	\textbf{RF 10} & Recibir consulta & El microservicio MAS debe recibir la consulta para iniciar el flujo de agentes. & Media & Alta \\ 
	\textbf{RF 11} & Analizar consulta & El Agente Moderador debe analizar la consulta para identificar términos clave relevantes. & Alta & Alta \\ 
	\textbf{RF 12} & Determinar acción & El Agente Moderador debe interpretar el propósito principal de la consulta del usuario para determinar la próxima acción del sistema (información, análisis,gráficas estadísticas etc.). & Alta & Alta \\ 
	\textbf{RF 13} & Generar embeddings & El Agente Recuperador debe convertir las palabras clave en representaciones vectoriales (embeddings) adecuadas para la búsqueda semántica. & Alta & Alta \\ 
	\textbf{RF 14} & Consultar base de datos & El Agente Recuperador debe buscar y obtener fragmentos de texto relevantes del \textit{Diario de la Marina} desde la base de datos vectorial, basándose en los embeddings. & Alta & Alta \\
	\textbf{RF 15} & Consultar CSV & El Agente PandasAi debe buscar y obtener fragmentos de texto relevantes del \textit{Diario de la Marina} desde la base de datos en formato csv. & Alta & Alta \\ 
	\textbf{RF 16} & Contextualizar respuesta & El Agente Contextualizador debe generar una respuesta coherente en lenguaje natural, integrando la información recuperada y añadiendo contexto histórico si es pertinente. & Alta & Alta \\ 
	\textbf{RF 17} & Identificar estadísticas & El Agente Contextualizador debe determinar si la consulta o los datos recuperados justifican la creación de una representación gráfica. & Media & Media \\ 
	\textbf{RF 18} & Formular instrucciones para gráfico & Si se requiere un gráfico, el Agente Contextualizador debe generar las especificaciones (tipo de gráfico, datos a usar) para el Agente PandasAi. & Alta & Alta \\ 
	\textbf{RF 19} & Generar script & El Agente PandasAI debe crear un script ejecutable que produzca la visualización solicitada a partir de los datos y especificaciones. & Alta & Alta \\ 
	\textbf{RF 20} & Validar script & El Agente PandasAi debe verificar que el script generado es sintácticamente correcto y no contiene errores obvios antes de ejecutarlo. & Alta & Alta \\ 
	\textbf{RF 21} & Generar imagen & El Agente PandasAi debe ejecutar el script validado para generar la visualización como un archivo de imagen (e.g., PNG, JPG). & Alta & Alta \\ 
	\textbf{RF 22} & Combinar resultados & El Agente Contextualizador (o Moderador) debe combinar la respuesta textual y/o la imagen generada en una estructura de respuesta unificada. & Alta & Alta \\ 
	\textbf{RF 23} & Validar respuesta & El Agente de Validación debe revisar la respuesta preliminar para asegurar su precisión, coherencia con la consulta y ausencia de información errónea. & Alta & Alta \\  
	\textbf{RF 24} & Enviar respuesta & El microservicio MAS debe enviar la respuesta final ensamblada al backend a través de su API. & Media & Alta \\ 
	% --- Flujo de Respuesta Backend y Frontend ---
	\textbf{RF 25} & Visualizar respuesta & El backend debe enviar la respuesta (texto y/o referencia a la imagen) al frontend para su visualización. & Media & Alta \\ 
	
\end{longtable}

\subsection{Requisitos no funcionales}

Los requisitos no funcionales son aquellos que no se refieren directamente a las funciones
específicas que proporciona el sistema, sino a las propiedades de este como fiabilidad,
tiempo de respuesta y la capacidad de almacenamiento. Incluyen además restricciones de
tiempo, sobre el proceso de desarrollo y estándares~\cite{sommerville2011software}.
A continuación, se definen los requisitos no funcionales que debe cumplir la aplicación
basándose en los establecido por las normas ISO 25000 Calidad del Producto de Software,
específicamente la ISO/IEC 25010 que define las características de calidad que se tienen
en cuenta al evaluar las propiedades de un producto de software~\cite{iso25010-2023}.
A continuación, se listan los requisitos no funcionales identificados:

\begin{longtable}{@{}p{2cm}p{12cm}@{}}
	\caption{Requisitos No Funcionales (RNF)} \label{tab:rnf_iso25010} \\
	\toprule
	\textbf{ID} & \textbf{Descripción} \\
	\midrule
	\endfirsthead
	
	\caption[]{Requisitos No Funcionales (RNF) (continuación)} \\
	\toprule
	\textbf{ID} & \textbf{Descripción} \\
	\midrule
	\endhead
	
	\bottomrule
	\endfoot
	
	% --- RENDIMIENTO ---
	\multicolumn{2}{@{}l}{\textbf{RNF 1: Rendimiento}} \\[0.5ex]
	RNF 1.1 & Tiempo de respuesta para consultas simples (búsqueda de barcos por nombre o fecha) no superior a los 30 segundos para 20 usuarios concurrentes. \\
	RNF 1.2 & Tiempo de respuesta para consultas complejas con generación de gráficos no superior a un minuto para 20 usuarios concurrentes. \\
	\addlinespace
	
	% --- SEGURIDAD ---
	\multicolumn{2}{@{}l}{\textbf{RNF 2: Seguridad}} \\[0.5ex]
	RNF 2.1 & Almacenamiento seguro de contraseñas de usuarios \\
	RNF 2.2 & Rastreabilidad de decisiones tomadas por los agentes IA \\
	RNF 2.3 & El acceso a la información debe estar restringido por usuario, contraseña.\\
	\addlinespace
	
	% --- USABILIDAD ---
	\multicolumn{2}{@{}l}{\textbf{RNF 3: Usabilidad}} \\[0.5ex]
	RNF 3.1 & Retroalimentación visual durante el procesamiento de consultas \\
	RNF 3.2 & Mensajes de error claros y comprensibles para usuarios \\
	\addlinespace
	
	% --- FIABILIDAD ---
	\multicolumn{2}{@{}l}{\textbf{RNF 4: Fiabilidad}} \\[0.5ex]
	RNF 4.1 & Alta disponibilidad del sistema completo \\
	RNF 4.2 & El sistema debe ser tolerante a fallos, y mostrar solo la información necesaria para orientar al usuario\\
	\addlinespace
		
	% --- MANTENIBILIDAD ---
	\multicolumn{2}{@{}l}{\textbf{RNF 5: Mantenibilidad}} \\[0.5ex]
	RNF 5.1 & El software estará bien documentado de forma tal que el tiempo de mantenimiento
	sea mínimo en caso de necesitarse. \\
	RNF 5.2 & Se debe hacer uso de los estándares de codificación definidos para el sistema multiagente \\
	RNF 5.3 & Gestión controlada de dependencias \\
	\addlinespace
	
\end{longtable}


\subsection{Historias de usuarios}

Las historias de usuario (HU) son descripciones breves y simples de los requerimientos de un cliente o usuario, que facilitan la comunicación con los desarrolladores del proyecto. Estas historias permiten expresar las expectativas y necesidades de los usuarios de una manera clara y comprensible, evitando ambigüedades y malentendidos que podrían llevar a pérdidas de tiempo y recursos \cite{menzinsky2018historias}.

Se realiza una HU por cada RF del componente, a continuación, se mostrarán las HU más relevantes. Para consultar el resto de las historias de usuario, se remite al \textbf{Anexo A}, donde se presentan de forma completa y organizada. Estas historias complementan la visión general del sistema y permiten una comprensión más exhaustiva de los requerimientos funcionales detallados.

% ==========================================
% Funcionalidad Principal del Chat y MAS
% ==========================================

\begin{userstory}[hu:08]
	\storyname{Interactuar con el chat activo (Enviar Consulta y Ver Respuesta)}
	\storyuser{Usuario autenticado}
	\storyiter{1} % Iteración ajustada
	\storypriority{Alta} % Basado en RF9 (Input), RF10, RF26, RF27, RF28
	\storyrisk{Medio} % Riesgo en la comunicación FE-BE-MAS
	\storypoints{2 semanas} % Estimación ajustada (cubre FE y BE básico)
	\storyprogrammer{Daniel Rojas Grass}
	\storydescription{
		Como usuario autenticado, quiero poder escribir una consulta en lenguaje natural en la interfaz de chat, enviarla al sistema, y ver tanto mi consulta como la respuesta del sistema (texto y/o imagen) mostradas de forma clara y ordenada en el área de diálogo de la conversación activa. (Corresponde principalmente a RF9-Input, RF10, RF26, RF27, RF28)
		
		\textbf{Precondiciones:}
		\begin{itemize}
			\item El usuario tiene una sesión activa.
			\item El usuario tiene una conversación activa (nueva HU:04 o cargada HU:06).
			\item El \textit{frontend}, el \textit{backend} y el microservicio MAS están operativos y pueden comunicarse entre sí.
		\end{itemize}
		
		\textbf{Flujo de acción:}
		\begin{enumerate}
			\item Usuario escribe una consulta en el campo de texto del chat.
			\item Usuario envía la consulta (click en botón 'Enviar' o presiona 'Enter').
			\item El \textit{frontend} muestra inmediatamente la consulta del usuario en el área de diálogo (marcada como del usuario).
			\item El \textit{frontend} envía la consulta y el ID de la conversación activa (si existe) al \textit{backend}.
			\item (Flujo cubierto en HU:09 y HU:10) El \textit{backend} recibe la consulta, la envía al microservicio MAS para procesamiento, recibe la respuesta del MAS (texto y/o referencia a imagen) y la almacena en la BD asociada a la conversación (RF9-Persistencia).
			\item El \textit{backend} envía la respuesta procesada (texto y/o URL/datos de imagen) al frontend.
			\item El \textit{frontend} recibe la respuesta del backend.
			\item El \textit{frontend} muestra la respuesta del sistema (texto y/o imagen) en el área de diálogo (marcada como del sistema).
		\end{enumerate}
	}
	\storyobservation{
		La interfaz debe indicar visualmente cuándo el sistema está procesando la respuesta. El manejo de errores (fallos de red, errores del MAS) debe ser robusto, mostrando mensajes apropiados al usuario. La visualización del chat debe permitir scroll para ver mensajes antiguos.
	}
	\storyinterface{Interfaz principal del Chat en el sitio web mostrando diálogo:
		\par\medskip % Añade un pequeño espacio vertical
		\begin{center} % Para centrar la imagen
			\includegraphics[width=0.6\textwidth]{images/chat.PNG} % Imagen del ejemplo
		\end{center}
		\medskip
	}
	
\end{userstory}

\begin{userstory}[hu:09]
	\storyname{Obtener respuesta textual relevante del MAS}
	\storyuser{Usuario autenticado (indirectamente, a través del sistema)}
	\storyiter{2} % Iteración estimada
	\storypriority{Alta} % Basado en RF11-16, RF22-25
	\storyrisk{Alto} % Riesgo asociado a la calidad/precisión de la IA
	\storypoints{4 semanas} % Estimación basada en ejemplo (complejidad alta)
	\storyprogrammer{Daniel Rojas Grass}
	\storydescription{
		Como sistema (actuando en nombre del usuario), quiero que el microservicio MAS procese una consulta recibida del \textit{backend}, la analice (palabras clave, intención), recupere información relevante del Diario de la Marina desde la BD vectorial, sintetice una respuesta textual coherente y contextualizada, la valide y la devuelva al \textit{backend}, para que el usuario final reciba información precisa. (Corresponde al flujo interno del MAS: RF11, RF12, RF13, RF14, RF15, RF16, RF22, RF23, RF24, RF25)
		
		\textbf{Precondiciones:}
		\begin{itemize}
			\item El microservicio MAS (FastAPI) está operativo.
			\item Todos los agentes internos (Moderador, Pandasai, Contextualizador, Validación) están implementados y disponibles.
			\item La base de datos vectorial (Faiss) está cargada con los datos del *Diario de la Marina* y es accesible.
			\item El csv con los datos del \textit{Diario de la Marina} esta cargado y es accesible.
			\item El LLM subyacente está configurado y accesible.
			\item El MAS recibe una solicitud válida del \textit{backend} a través de su API.
		\end{itemize}
		
		\textbf{Flujo de acción (interno del MAS):}
		\begin{enumerate}
			\item MAS recibe la consulta vía API (RF11).
			\item Agente Moderador extrae palabras clave y determina intención (RF12, RF13).
			\item Agente Recuperador genera \textit{embeddings} y busca en BD vectorial (RF14, RF15).
			\item Agente PandasAi genera tanto gráficas como búsquedas de análisis en el csv.
			\item Agente Contextualizador recibe fragmentos relevantes y genera respuesta textual inicial, añadiendo contexto (RF16).
			\item (Si no se requiere gráfico) Agente Contextualizador (o Moderador) prepara la respuesta textual preliminar (RF22).
			\item Agente Validación revisa coherencia, relevancia y posible toxicidad/error (RF23).
			\item Agente Moderador ensambla la respuesta textual validada en formato JSON (RF24).
			\item MAS retorna la respuesta JSON al backend DRF (RF25).
		\end{enumerate}
	}
	\storyobservation{
		La calidad de los embeddings y la estrategia de recuperación son críticas (RF14, RF15). La capacidad del LLM para sintetizar y contextualizar sin "alucinar" es fundamental (RF16). El agente de validación es clave para la fiabilidad (RF23). La latencia del proceso completo debe ser aceptable (considerar RNF).
	}
	\storyinterface{[N/A - Proceso interno del Microservicio MAS]}
	
\end{userstory}

\begin{userstory}[hu:10]
	\storyname{Recibir visualización gráfica cuando sea pertinente}
	\storyuser{Usuario autenticado}
	\storyiter{3} % Iteración estimada
	\storypriority{Media} % Basado en RF17-21
	\storyrisk{Moderado} % Riesgo en la generación y validación del script/gráfico
	\storypoints{2 semanas} % Estimación basada en ejemplo (complejidad alta)
	\storyprogrammer{Daniel Rojas Grass}
	\storydescription{
		Como usuario autenticado, quiero que cuando mi consulta o la información recuperada sugieran la necesidad de una visualización (e.g., análisis de tendencias, comparación de datos), el sistema (específicamente el MAS) genere un gráfico apropiado (e.g., barras, líneas) y me lo presente como una imagen dentro de la respuesta del chat, junto con el texto explicativo, para facilitar mi comprensión. (Corresponde principalmente a RF17, RF18, RF19, RF20, RF21, RF22-integración, RF28-visualización)
		
		\textbf{Precondiciones:}
		\begin{itemize}
			\item El flujo de HU:09 está en progreso.
			\item El Agente Contextualizador identifica que la consulta/datos justifican un gráfico (RF17).
			\item Los datos necesarios para el gráfico están disponibles y estructurados.
			\item El Agente PandasAi esta disponible para la generación de gráficas.
		\end{itemize}
		
		\textbf{Flujo de acción (continuación de HU:09):}
		\begin{enumerate}
			\item Agente Contextualizador determina necesidad de gráfico y formula instrucciones (tipo, datos) (RF17, RF18).
			\item Agente Contextualizador invoca al Agente PandasAi con las instrucciones.
			\item Agente PandasAi genera el script de Pandas para crear el gráfico (RF19).
			\item Agente PandasAi valida la sintaxis y lógica básica del script (RF20).
			\item Agente PandasAi ejecuta el script validado para generar la imagen del gráfico (e.g., PNG) (RF21).
			\item Agente PandasAi devuelve la imagen (o una referencia a ella) al Agente Contextualizador/Moderador.
			\item Agente Contextualizador/Moderador integra la imagen junto con la respuesta textual en la estructura preliminar (RF22).
			\item (Continúa flujo de HU:09) Validación (RF23), Ensamblaje (RF24), Retorno al \textit{backend} (RF25).
			\item (Flujo de HU:08) \textit{Backend} envía URL/datos de imagen al \textit{frontend}.
			\item \textit{Frontend} muestra la imagen recibida dentro del chat (RF28).
		\end{enumerate}
	}
	\storyobservation{
		La detección de la necesidad de un gráfico debe ser fiable (RF17). La generación del script debe ser segura (evitar ejecución de código arbitrario). La validación del script (RF20) es importante para evitar errores en tiempo de ejecución. Los gráficos deben ser claros y legibles. Considerar formatos de imagen web-friendly (PNG, JPG, SVG).
	}
	\storyinterface{Visualización de imagen (gráfico) dentro del Chat en el sitio web, junto al texto explicativo.}
	
\end{userstory}

Las historias de usuario presentadas detallan los requerimientos funcionales del sistema desde la perspectiva del usuario, abarcando desde la gestión de usuarios y autenticación hasta la interacción con el chat y la generación de respuestas textuales y gráficas por parte del microservicio MAS. Estas historias no solo especifican el comportamiento esperado del sistema, sino que también establecen una base clara para el diseño y desarrollo del software, asegurando que las necesidades del usuario se traduzcan en funcionalidades concretas.



\subsection{Tarjetas CRC}

Las tarjetas CRC (Clase-Responsabilidad-Colaboración) son una herramienta de diseño de software orientado a objetos, creada por Kent Beck y Ward Cunningham. Estas tarjetas se utilizan para identificar las clases, sus responsabilidades y cómo colaboran con otras clases para cumplir tareas específicas en un sistema \cite{BeckCunningham}. La representación del sistema multiagente y su interfaz gráfica mediante tarjetas CRC permite estructurar de forma clara y concisa las clases que lo componen, sus responsabilidades específicas y las colaboraciones necesarias para cumplir sus objetivos. Esta técnica facilita la comprensión del comportamiento de cada agente dentro del sistema —como el coordinador, el buscador de información o el generador de respuestas—, promoviendo un diseño orientado a objetos coherente, reutilizable y fácil de mantener. Además, al emplearse en etapas tempranas del desarrollo, las tarjetas CRC fortalecen la comunicación entre desarrolladores y respaldan la validación del modelo antes de su implementación definitiva.

\begin{longtable}{|l|l|}
	\caption{Tarjeta CRC: Agente Moderador} \label{tablacrc1} \\
	
	\hline
	\multicolumn{2}{|c|}{\textbf{Tarjeta CRC}} \\
	\hline
	\textbf{Clase} & \textbf{Agente Moderador} \\
	\hline
	\endfirsthead
	
	\hline
	\textbf{Responsabilidad} & \textbf{Colaboración} \\
	\hline
	\endhead
	
	\hline
	\multicolumn{2}{|r|}{Continúa en la próxima página} \\
	\hline
	\endfoot
	
	\hline
	\endlastfoot
	
	\parbox[t]{0.45\linewidth}{\textbf{Responsabilidades:} \\ 
		Recibir la consulta del usuario \\ 
		Extraer palabras clave e identificar la intención \\ 
		Coordinar el flujo de información entre agentes \\ 
		Ensamblar y entregar la respuesta final al usuario} 
	& 
	\parbox[t]{0.45\linewidth}{\textbf{Colaboración:} \\
		Agente PandasAi \\ 
		Agente recuperador de información (FAISS)\\
		Agente Contextualizador \\ 
		Agente de Validación}
\end{longtable}


\begin{longtable}{|l|l|}
	\caption{Tarjeta CRC: Agente recuperador de información (FAISS) } \label{tablacrc2} \\
	
	\hline
	\multicolumn{2}{|c|}{\textbf{Tarjeta CRC}} \\
	\hline
	\textbf{Clase} & \textbf{Agente recuperador de información (FAISS)} \\
	\hline
	\endfirsthead
	
	\hline
	\textbf{Responsabilidad} & \textbf{Colaboración} \\
	\hline
	\endhead
	
	\hline
	\multicolumn{2}{|r|}{Continúa en la próxima página} \\
	\hline
	\endfoot
	
	\hline
	\endlastfoot
	
	\parbox[t]{0.45\linewidth}{\textbf{Responsabilidades:} \\ 
		Convertir las palabras clave en \textit{embeddings} \\ 
		Consultar la base de datos vectorial para recuperar la información relevante \\ 
		Devolver los resultados al Agente Moderador} 
	& 
	\parbox[t]{0.45\linewidth}{\textbf{Colaboración:} \\
		Agente Moderador \\ 
		Base de datos vectorial}
\end{longtable}

\begin{longtable}{|l|l|}
	\caption{Tarjeta CRC: Agente Contextualizador} \label{tablacrc3} \\
	
	\hline
	\multicolumn{2}{|c|}{\textbf{Tarjeta CRC}} \\
	\hline
	\textbf{Clase} & \textbf{Agente Contextualizador} \\
	\hline
	\endfirsthead
	
	\hline
	\textbf{Responsabilidad} & \textbf{Colaboración} \\
	\hline
	\endhead
	
	\hline
	\multicolumn{2}{|r|}{Continúa en la próxima página} \\
	\hline
	\endfoot
	
	\hline
	\endlastfoot
	
	\parbox[t]{0.45\linewidth}{\textbf{Responsabilidades:} \\ 
		Recibir la información relevante y la intención del usuario \\ 
		Generar indicaciones de estadísticas o contextualizar la información según la petición del usuario \\ 
		Enviar la información procesada al Agente Moderador} 
	& 
	\parbox[t]{0.45\linewidth}{\textbf{Colaboración:} \\
		Agente Moderador \\ 
		Agente PandasAi}
\end{longtable}


\begin{longtable}{|l|l|}
	\caption{Tarjeta CRC: Agente de Validación} \label{tablacrc4} \\
	
	\hline
	\multicolumn{2}{|c|}{\textbf{Tarjeta CRC}} \\
	\hline
	\textbf{Clase} & \textbf{Agente de Validación} \\
	\hline
	\endfirsthead
	
	\hline
	\textbf{Responsabilidad} & \textbf{Colaboración} \\
	\hline
	\endhead
	
	\hline
	\multicolumn{2}{|r|}{Continúa en la próxima página} \\
	\hline
	\endfoot
	
	\hline
	\endlastfoot
	
	\parbox[t]{0.45\linewidth}{\textbf{Responsabilidades:} \\ 
		Revisar la coherencia de la respuesta generada por el sistema \\ 
		Comparar la respuesta con la entrada original \\ 
		Validar o rechazar la respuesta antes de enviarla al usuario} 
	& 
	\parbox[t]{0.45\linewidth}{\textbf{Colaboración:} \\
		Agente Moderador \\ 
		Agente Contextualizador}
\end{longtable}

\begin{longtable}{|l|l|}
	\caption{Tarjeta CRC: Agente PandasAi} \label{tablacrc5} \\
	
	\hline
	\multicolumn{2}{|c|}{\textbf{Tarjeta CRC}} \\
	\hline
	\textbf{Clase} & \textbf{Agente PandasAi} \\
	\hline
	\endfirsthead
	
	\hline
	\textbf{Responsabilidad} & \textbf{Colaboración} \\
	\hline
	\endhead
	
	\hline
	\multicolumn{2}{|r|}{Continúa en la próxima página} \\
	\hline
	\endfoot
	
	\hline
	\endlastfoot
	
	\parbox[t]{0.45\linewidth}{\textbf{Responsabilidades:} \\ 
		Generar un \textit{script} de Pandas para representar estadísticas gráficas cuando sea necesario \\ 
		Validar el \textit{script} y asegurarse de que sea correcto \\ 
		Convertir el gráfico generado en una imagen que pueda ser integrada en la respuesta final\\
		Hacer consultas al CSV cargado para hacer análisis profundos de datos.} 
	& 
	\parbox[t]{0.45\linewidth}{\textbf{Colaboración:} \\
		Agente Contextualizador \\ 
		}
\end{longtable}

En este documento se incluyen ejemplos representativos de algunas de las tarjetas CRC más relevantes para el desarrollo de la solución propuesta. Sin embargo, debido a su extensión, se remite al \textit{Anexo B} para consultar el conjunto completo de las tarjetas CRC, donde se detallan todas las clases identificadas, sus responsabilidades y los colaboradores correspondientes.

\section{Diseño de la propuesta de solución}

\subsection{Diseño de la arquitectura}

La arquitectura propuesta adopta el estilo de \emph{microservicios}, en el cual la aplicación se descompone en un conjunto de servicios independientes, cada uno ejecutándose en su propio proceso y comunicándose mediante protocolos API REST~\cite{turn0search0,turn0search9}. Cada microservicio se orienta a una capacidad de negocio específica, lo que facilita la comprensión y el mantenimiento aislado de los componentes~\cite{turn0search6,turn0search0}. La modularidad inherente a este enfoque permite el despliegue automatizado e independiente de cada servicio, mejorando la agilidad operativa y reduciendo el tiempo de inactividad asociado a las actualizaciones \cite{turn0search8,turn0search3}. La escalabilidad horizontal se ve potenciada, dado que cada servicio puede replicarse de forma autónoma según la demanda, optimizando el uso de recursos y posibilitando un dimensionamiento granular \cite{turn0search3,turn1search0}. El desacoplamiento entre servicios incrementa la tolerancia a fallos, ya que la interrupción de un componente no compromete la disponibilidad del sistema global \cite{turn0search2,turn0search6}. La heterogeneidad tecnológica está plenamente soportada, puesto que cada microservicio puede implementarse con los lenguajes y frameworks más adecuados para su responsabilidad particular \cite{turn0search6,turn1search8}.

\begin{figure}[htbp] 
	\centering
	\includegraphics[width=1\textwidth]{images/Arquitectura.png} 
	\caption{Arquitectura de microservicios del sistema multiagente (Fuente: elaboración propia).}
	\label{fig:erquitectura_MAS}
\end{figure}

La presente configuración arquitectónica se materializa en tres componentes principales como se puede visualizar en la Figura \ref{fig:erquitectura_MAS}:
  
\begin{enumerate}
	\item Un frontend desarrollado con React, encargado de la interacción con el usuario.  
	\item Un backend construido con Django REST Framework, que actúa como orquestador de la lógica de negocio y gestor de peticiones.  
	\item Un microservicio especializado en procesamiento de lenguaje natural, implementado con FastAPI, que alberga el sistema multiagente conversacional.  
\end{enumerate}

Esta elección garantiza la independencia en el ciclo de vida de cada servicio, promoviendo la mantenibilidad y facilitando la incorporación de nuevas tecnologías o la sustitución de componentes sin afectar al sistema global \cite{turn0search1,turn1search4}.

\subsection{Diseño del modelo de datos}

\begin{figure}[htbp] 
	\centering
	\includegraphics[width=1\textwidth]{images/modelo.png} 
	\caption{Diseño del modelo de datos (Fuente: elaboración propia).}
	\label{fig:modelo_de_datos}
\end{figure}


El modelo de datos propuesto en la Figura \ref{fig:modelo_de_datos} articula de manera coherente la estructura relacional necesaria para soportar un sistema de chat en tiempo real~\cite{Codd1970}. Se fundamenta en los principios del modelo relacional que establecen las bases para el almacenamiento y la recuperación eficiente de datos, garantizando la unicidad de las tuplas y la ausencia de redundancia innecesaria~\cite{Silberschatz2019}.\\
La definición de la entidad Usuario contempla un identificador único uid, atributos de autenticación y perfil, así como indicadores de estado y marcas temporales de creación y modificación, satisfaciendo los requerimientos de integridad y auditoría en el dominio de usuario~\cite{ElmasriNavathe2010}. La elección de tipos de datos VARCHAR para los campos característicos responde a la flexibilidad necesaria para posibles extensiones en los identificadores y datos personales~\cite{Date2003}.\\
La entidad Sala de chat permite agrupar conversaciones de manera lógica y filtrable~\cite{ConnollyBegg2014}. Al establecer una relación de uno a muchos con la entidad Usuario a través de la clave foránea registered\_by, se refuerza la trazabilidad de la creación y administración de espacios de comunicación~\cite{Harrington2015}.\\
La entidad Mensajes está diseñada para almacenar cada mensaje asociado a una sala de chat, con su propio identificador, estado de actividad, contenido textual, atributos de rol y metadatos adicionales que facilitan la moderación y la representación multimodal~\cite{Silberschatz2019}. La implementación de la clave foránea chat\_room garantiza la integridad referencial, evitando mensajes huérfanos y asegurando la coherencia de las interacciones~\cite{IBMReferential}.\\
El diseño enfatiza la normalización hasta la tercera forma normal, minimizando anomalías de actualización y duplicación de datos~\cite{Silberschatz2019}. Asimismo, la adopción de métodos ágiles para el modelado de datos respalda iteraciones rápidas y adaptativas durante el ciclo de vida del desarrollo~\cite{Ambler2003}. La adherencia al estándar SQL definido en ISO/IEC 9075 certifica la interoperabilidad en entornos heterogéneos de bases de datos~\cite{ISO9075}.Finalmente, la consideración de criterios de escalabilidad y rendimiento se inspira en lecciones históricas sobre arquitecturas de sistemas de datos~\cite{Stonebraker2005}.


\section{Implementación}

\subsection{Estándares de codificación en Python y JavaScript}

Los estándares de codificación son esenciales para garantizar la mantenibilidad, colaboración y calidad del software~\cite{PEP8}. En este epígrafe se analiza comparativamente las convenciones principales para Python y JavaScript, destacando sus similitudes y diferencias fundamentales.

\subsubsection{Estándares para Python}

\begin{enumerate}
	\item \textbf{Convenciones de Estilo}: Python cuenta con una guía oficial de estilo denominada PEP 8~\cite{PEP8}, que establece directrices precisas para la escritura de código. La indentación de 4 espacios (prohibiendo el uso de tabuladores) y el límite de 79 caracteres por línea son dos de sus reglas más características. El espaciado alrededor de operadores y la organización de imports en tres grupos (bibliotecas estándar, terceros y locales) promueven la consistencia visual. La nomenclatura sigue patrones específicos: \texttt{snake\_case} para funciones y variables, \texttt{PascalCase} para clases, y \texttt{MAYÚSCULAS} para constantes. La documentación mediante docstrings (siguiendo PEP 257~\cite{PEP257}) facilita la generación automática de documentación técnica.
	\item \textbf{Herramientas y Prácticas}: El ecosistema Python ofrece herramientas como \texttt{flake8} para verificación de estilo y \texttt{black} para formateo automático. El uso de \textit{type hints} (desde Python 3.5) mejora la seguridad de tipos en proyectos complejos. El manejo de excepciones debe ser explícito, evitando cláusulas \texttt{except} genéricas que puedan ocultar errores.
\end{enumerate}

\subsubsection{Estándares para JavaScript}

\begin{enumerate}
	\item \textbf{Guías de Referencia}: JavaScript carece de un estándar oficial único, pero guías como el Airbnb JavaScript Style Guide~\cite{AirbnbJS} y Google JavaScript Style Guide~\cite{GoogleJS} se han consolidado como referencias. Estas enfatizan el uso de ES6+, con preferencia por \texttt{const/let} sobre \texttt{var}, y arrow functions para callbacks.\\
	La convención de nombres utiliza \texttt{camelCase} para variables/funciones y \texttt{PascalCase} para clases. Los literales de plantilla (template strings) son preferidos sobre concatenación manual, y el punto y coma sigue siendo opcional pero debe usarse consistentemente.
	\item \textbf{Ecosistema Moderno}: Herramientas como ESLint (configurable con reglas específicas) y Prettier automatizan la aplicación de estándares. El sistema de módulos ES6 (\texttt{import/export}) ha reemplazado ampliamente a \texttt{require} en proyectos nuevos. Para proyectos complejos, TypeScript ofrece tipado estático, reduciendo errores en tiempo de ejecución~\cite{TypeScript}.
\end{enumerate}

Ambos lenguajes comparten principios fundamentales: modularización del código, documentación clara, y uso de linters. Python muestra mayor uniformidad gracias a PEP 8, mientras JavaScript permite más flexibilidad mediante guías configurables. En rendimiento, JavaScript prioriza la compatibilidad con navegadores, mientras Python enfatiza la legibilidad como principio filosófico~\cite{PEP20}. La adopción de estándares debe adaptarse al contexto del proyecto y equipo, utilizando herramientas automatizadas para garantizar cumplimiento. Tanto Python como JavaScript han desarrollado ecosistemas maduros que, cuando se usan consistentemente, elevan sustancialmente la calidad del código.

\subsection{Patrones de diseño}
\label{sec:patrones_diseno}

Los patrones de diseño de software representan soluciones probadas y estandarizadas para problemas recurrentes en el desarrollo, encapsulando mejores prácticas y promoviendo la creación de software modular, legible, flexible y robusto \cite{gavilanez2022analisis}. En el desarrollo de esta aplicación web y su microservicio asociado, se han aplicado conscientemente varios patrones para abordar los desafíos inherentes a su arquitectura distribuida y su flujo de trabajo basado en IA. A continuación, se analizan algunos de los patrones de diseño a implementar en el sistema multiagente conversacional y en el \textit{backend} (desarrollado con Django REST Framework), destacando su relevancia y su impacto en el procesamiento de datos históricos del \textit{Diario de la Marina}.

\textbf{Patrón Singleton} se utiliza para gestionar recursos críticos compartidos en el sistema multiagente, asegurando que componentes como la base de datos históricos, el índice FAISS para la base de datos vectorial, y los agentes tengan una única instancia en toda la aplicación. Este patrón es fundamental para optimizar el uso de recursos, ya que evita la creación redundante de instancias pesadas, lo que podría impactar negativamente el rendimiento del sistema, especialmente al procesar grandes volúmenes de datos históricos del \textit{Diario de la Marina}. La Figura \ref{fig:codigo_singleton} muestra un extracto de este código, evidenciando la implementación del patrón.\\
La importancia del patrón \textit{Singleton} radica en su capacidad para centralizar el acceso a recursos compartidos, reduciendo la sobrecarga computacional y asegurando consistencia en los datos procesados. Esto es particularmente relevante en un sistema multiagente, donde múltiples agentes (como \texttt{AgenteRecuperador} y \texttt{AgentePandasAi}) necesitan acceder a las mismas estructuras de datos para generar respuestas textuales o gráficas.

\begin{figure}[htbp]
	\centering
	\includegraphics[width=0.8\textwidth]{images/singleton.PNG}
	\caption{Extracto de código que implementa el patrón (Fuente: elaboración propia). \textit{Singleton}.}
	\label{fig:codigo_singleton}
\end{figure}

\textbf{Patrón Cadena de responsabilidad} permite que el sistema multiagente implemente un flujo de trabajo estructurado, que orqueste la ejecución de agentes a través de un grafo de estados. Este patrón permite que cada agente (como el moderador, el contextualizador, el ejecutor PandasAI, y el validador) procese y modifique un estado compartido, pasando el control al siguiente agente en la cadena. Esta estructura es esencial para manejar consultas complejas que requieren el procesamiento conjunto de datos estructurados (del archivo CSV) y no estructurados (de la base vectorial), como se describe en la Figura \ref{fig:codigo_workflow}.\\ 
La relevancia de este patrón radica en su capacidad para modularizar el flujo de procesamiento, permitiendo que cada agente se especialice en una tarea específica. Esto no solo facilita el mantenimiento y la extensibilidad del sistema, sino que también asegura que las consultas se procesen de manera eficiente, cumpliendo con los requisitos funcionales relacionados con la generación de respuestas.

\begin{figure}[htbp]
	\centering
	\includegraphics[width=0.8\textwidth]{images/cadena.PNG}
	\caption{Extracto de código que implementa el patrón \textit{Cadena de responsabilidad} (Fuente: elaboración propia).}
	\label{fig:codigo_workflow}
\end{figure}

\textbf{Patrón Plantilla Abstracta} permite estandarizar comportamientos comunes en modelos, serializadores y paginaciones. Este patrón permite definir una estructura general que puede ser heredada y personalizada por clases específicas, promoviendo la reutilización de código y facilitando el mantenimiento del sistema. La Figura \ref{fig:codigo_template_method} muestra un extracto de este código.\\
La importancia del patrón \textit{Plantilla Abstracta} en este contexto radica en su capacidad para reducir la duplicación de código y garantizar consistencia en la estructura de los datos manejados por el \textit{backend}. Esto es crucial en un sistema que orquesta múltiples microservicios, ya que asegura que las respuestas enviadas al \textit{frontend} sean uniformes y que los modelos de datos sean fácilmente extensibles para futuras funcionalidades.

\begin{figure}[htbp]
	\centering
	\includegraphics[width=0.8\textwidth]{images/codigo_template_method.png}
	\caption{Extracto de código que implementa el patrón \textit{Plantilla Abstracta} (Fuente: elaboración propia).}
	\label{fig:codigo_template_method}
\end{figure}

\textbf{Patrón Fachada} permite simplificar la interacción con subsistemas complejos, encapsulando la lógica de manejo de mensajes y respuestas en clases específicas. Este patrón permite abstraer operaciones complejas, como el formateo de mensajes o la gestión de respuestas de un modelo de lenguaje, en una interfaz sencilla que puede ser utilizada por otras partes del sistema. La Figura \ref{fig:codigo_facade} muestra ejemplo de la implementación de este patrón.\\
La relevancia del patrón \textit{Facade/Service} radica en su capacidad para reducir la complejidad del backend, permitiendo que los controladores de Django REST Framework se centren en la lógica de negocio mientras las clases de servicio manejan las operaciones más técnicas. Esto mejora la modularidad y facilita la integración con el sistema multiagente, asegurando que las respuestas generadas por el Microservicio MAS se procesen y entreguen al frontend de manera eficiente.

\begin{figure}[htbp]
	\centering
	\includegraphics[width=0.8\textwidth]{images/fachada.PNG}
	\caption{Extracto de código que implementa el patrón \textit{Fachada} (Fuente: elaboración propia).}
	\label{fig:codigo_facade}
\end{figure}

El uso de estos patrones de diseño en el sistema multiagente y el backend no solo asegura una implementación robusta y escalable, sino que también facilita la transformación de datos históricos del \textit{Diario de la Marina} en conocimiento estructurado, cumpliendo con los objetivos del proyecto.



\subsection{Interfaz Principal del Sistema}

La figura \ref{fig:interfaz} muestra la interfaz principal del sistema multiagente, diseñado para permitir la interacción en lenguaje natural con datos históricos del \textit{Diario de la Marina}. A continuación, se detalla cada uno de los componentes numerados de la interfaz:

\begin{figure}[h!]
	\centering
	\includegraphics[width=1\textwidth]{images/interfaz1.png}
	\caption{Interfaz principal del chat con el sistema multiagente (Fuente: elaboración propia)}
	\label{fig:interfaz}
\end{figure}

\begin{enumerate}[label=\textbf{\arabic*.}]
	\item \textbf{Botón de Nuevo Chat}: ubicado en la parte superior izquierda, este botón permite al usuario iniciar una nueva conversación. Al activarse, se borra la conversación actual y se prepara el sistema para recibir una nueva consulta.
	
	\item \textbf{Panel de Historial de Conversaciones}: muestra una lista cronológica de conversaciones anteriores. Cada elemento incluye el título de la conversación (asignado automáticamente o manualmente) y un menú contextual para acciones como renombrar o eliminar. Esta sección permite retomar diálogos previos de forma eficiente.
	
	\item \textbf{Información del Usuario}: en la parte inferior del panel lateral se muestra el correo electrónico del usuario autenticado. Incluye un botón para copiar fácilmente la dirección, facilitando su reutilización o verificación de sesión.
	
	\item \textbf{Área de Entrada de Consulta}: permite al usuario redactar y enviar preguntas o mensajes. El campo de texto cuenta con una sugerencia que invita a interactuar: \textit{¿Qué quieres saber hoy?}. Incluye un botón de envío que ejecuta la solicitud hacia el sistema conversacional.
	
\end{enumerate}

\subsection{Diagrama de despliege}

Los diagramas de despliegue muestran cómo los componentes de software se despliegan
físicamente en los procesadores; es decir, el diagrama de despliegue muestra el hardware
y el software en el sistema, así como el middleware~\footnote{Middleware, también conocido como lógica de intercambio de información entre aplicaciones o agente intermedio, es un sistema de software que ofrece servicios y funciones comunes para las aplicaciones.} usado para conectar los diferentes
componentes en el sistema. En esencia, los diagramas de despliegue se pueden considerar
como una forma de definir y documentar el entorno objetivo~\cite{sommerville2011software}.\\
A continuación, la figura~\ref{fig:despliege}. muestra el diagrama correspondiente al sistema propuesto.\\
\textbf{Nodos:} elementos de procesamiento con al menos un procesador, memoria, y posiblemente otros dispositivos.\\
\textbf{Dispositivos:} nodos estereotipados sin capacidad de procesamiento en el nivel de abstracción que se modela.\\
\textbf{Conectores:} expresa el tipo de conector o protocolo utilizado entre el resto de los elementos del modelo.\\

\begin{figure}[h!]
	\centering
	\includegraphics[width=1\textwidth]{images/despliege.drawio.png}
	\caption{Representación del modelo de despliegue. (Fuente: elaboración propia).}
	\label{fig:despliege}
\end{figure}

\textbf{Dispositivo del cliente}: Se refiere a el conjunto de todos los clientes que consumirán el software desde sus computadoras. La máquina cliente necesita de muy pocas
prestaciones; teniendo un navegador web (Chrome, Mozilla Firefox, Internet Explorer,Opera), una RAM mínima de 4 GB, una tarjeta de red y un procesador mínimo de 3.3 GB podrá acceder al sistema y realizar las operaciones necesarias\\
\textbf{Servidor Web}: Representa el servidor que se comunica con la PC Cliente mediante el protocolo HTTPS y además realiza peticiones al servidor de Bases de Datos mediante el protocolo TCP/IP, es el encargado de la presentación del repositorio, debe estar compuesto de 32 GB de RAM, 1 TB de almacenamiento, un procesador de 3.3 GHz o superior, una tarjeta de red, servidor web Nginx o Treafic con Docker.\\
\textbf{Servidor de BD}: Elemento de cómputo, dedicado a almacenar y proveer datos necesarios para el funcionamiento de la aplicación web. es el encargado de almacenar la información generada del
sistema, para el correcto funcionamiento del repositorio es necesario que posea PostgreSQL, una RAM mínima de 4 GB, un procesador mínimo de 3.3 GHz y un disco duro de 1 TB.\\

\section*{Conclusiones del capítulo}
\addcontentsline{toc}{section}{Conclusiones}

El capítulo 2 presenta un enfoque metodológico y técnico para abordar la interpretación de datos históricos no estructurados del Diario de la Marina, integrando metodologías ágiles, arquitecturas multiagente y tecnologías emergentes. La aplicación de Extreme Programming permitió adaptar el desarrollo a las necesidades identificadas en usuarios especializados, como historiadores, priorizando la corrección de errores de OCR y la contextualización de respuestas. La arquitectura propuesta, basada en microservicios y agentes especializados (Moderador, Recuperador, Contextualizador), mostró potencial para gestionar la complejidad de los datos mediante la coordinación de tareas como la recuperación semántica con FAISS y la generación de visualizaciones dinámicas.

La selección de herramientas como LangChain, modelos de embeddings y bases de datos vectoriales contribuyó a optimizar la precisión en el procesamiento de consultas, mientras que patrones de diseño como Singleton y Cadena de Responsabilidad reforzaron la mantenibilidad del sistema. La interfaz, diseñada para facilitar la interacción con usuarios no técnicos, combinó un historial de conversaciones y visualizaciones gráficas, evidenciando la viabilidad de transformar datos históricos en conocimiento accesible.

Este enfoque sugiere un avance en el campo de la informática histórica, al proponer un marco replicable que integra sistemas multiagente con IA generativa, adaptable a otros contextos de patrimonio digital. Las decisiones técnicas, respaldadas por un proceso iterativo y centrado en el usuario, reflejan una solución equilibrada entre innovación y aplicabilidad práctica, sentando bases para futuras validaciones experimentales y ampliaciones del sistema.
 % Capítulo 2
	\chapter{Pruebas de software}
\label{chap:chapter3}


El presente capítulo tiene como objetivo validar el correcto funcionamiento del sistema multiagente conversacional desarrollado para interpretar y contextualizar los datos no estructurados del transporte marítimo recogidos en el \textit{Diario de la Marina}. A través de un conjunto de pruebas cuidadosamente diseñadas, se evalúa si el sistema cumple con los requisitos funcionales y no funcionales previamente definidos, así como con los criterios de calidad establecidos por el modelo ISO/IEC 25010.

Para ello, se ejecutan pruebas unitarias, de integración y funcionales sobre los distintos componentes del sistema, prestando especial atención al comportamiento del microservicio de inteligencia artificial encargado del procesamiento conversacional. Asimismo, se emplean métricas como el tiempo de respuesta, la coherencia semántica de las respuestas generadas y la capacidad de recuperación de información relevante.

Además, se incluyen pruebas empíricas orientadas a medir la usabilidad desde la perspectiva del usuario final, validando si la interfaz permite una interacción fluida y si el sistema entrega respuestas precisas y contextualizadas ante consultas históricas reales.

\section{Pruebas de software}

Las pruebas tratan de demostrar que un programa hace lo que se intenta que haga, así como descubrir defectos en el programa antes de usarlo. Al probar el software, se ejecuta un programa con datos artificiales. Hay que verificar los resultados de la prueba que se opera para buscar errores, anomalías o información de atributos no funcionales del
programa~\cite{sommerville2011software}. A fin de encontrar los errores del sistema y garantizar un nivel aceptable de calidad y confianza, se realizaron pruebas de software, de caja negra, tanto manuales como automatizadas, haciendo uso de las principales técnicas existentes, y aprovechando las pruebas automatizadas, apoyándose en la estrategia de pruebas recomendada por el CMS, Desarrollo dirigido por pruebas (\textit{Test Driver Development}, por sus siglas en inglés TDD).

\section{Estrategia de pruebas}

La elaboración de todo producto de software implica la posibilidad de introducción de errores que provocan fallos en el sistema desarrollado. Por este motivo debe existir una vía para garantizar su calidad y correcto funcionamiento. La realización de pruebas es una actividad que permite verificar el producto bajo ciertas condiciones y en base a los requerimientos identificados para la construcción del mismo, los resultados son observados y registrados para su corrección.

\begin{longtable}{|p{2.5cm}|p{3cm}|p{3cm}|p{6cm}|}
	\caption{Estrategia de pruebas} \label{tab:plan-pruebas} \\
	\hline
	\textbf{Tipos de pruebas} & \textbf{Método} & \textbf{Herramienta} & \textbf{Alcance} \\
	\hline
	\endfirsthead
	
	\hline
	\textbf{Prueba} & \textbf{Método} & \textbf{Herramienta} & \textbf{Alcance} \\
	\hline
	\endhead
	
	Unitaria & Caja blanca aplicando la técnica del camino básico & Módulo TestCase de Django para la realización de pruebas automatizadas & Se automatizarán pruebas para las unidades de código separadas por requisitos funcionales. \\
	\hline
	Funcional & Caja negra aplicando la técnica de particiones equivalentes & SeleniumIDE & Se probará el funcionamiento del 100\% de los requisitos. \\
	\hline
	Rendimiento & Pruebas de carga y estrés & Locust & Se aplicará sobre un entorno de pruebas con prestaciones similares a las de despliegue establecidas en los requisitos no funcionales. Se probará la aplicación con 20 usuarios concurrentes buscando tiempos de respuesta menores a 5 segundos. \\
	\hline
	Seguridad &  & \textit{Acunetix Web Vulnerability Scanner} 9.5 & Se aplicará para detectar vulnerabilidades:
	\begin{itemize}[left=0pt]
		\item Inyección SQL
		\item Programación Cross-Site (XSS)
		\item Ataques de fuerza bruta a las credenciales
		\item Redirecciones y reenvíos no validados
	\end{itemize} \\
	\hline
\end{longtable}

\section{Pruebas unitarias}

Las pruebas unitarias son una técnica de desarrollo de software que consiste en verificar el correcto funcionamiento de las unidades individuales más pequeñas de un programa, como funciones o métodos. Su objetivo principal es asegurar que cada parte del código funcione según lo esperado, facilitando la detección de errores en etapas tempranas del desarrollo. Las pruebas unitarias son una parte esencial de las metodologías ágiles, ya que permiten mantener la calidad del software a lo largo del proceso de desarrollo.Estas pruebas las ejecuta el desarrollador, cada vez que va probando fragmentos de código
o scripts para ver si todo funciona como se desea. Proporcionan un contrato escrito, que el fragmento de código debe satisfacer. El método utilizado para realizar este tipo de prueba se denomina caja
blanca~\cite{sommerville2011software}.

\subsection{Método de caja blanca}

Las pruebas de caja blanca intentan garantizar que:

\begin{itemize}
	\item Se ejecutan al menos una vez todos los caminos independientes de cada requisito funcional.
	\item Se utilizan las decisiones en su parte verdadera y en su parte falsa.
	\item Se ejecuten todos los bucles en sus límites.
	\item Se utilizan todas las estructuras de datos internas.
\end{itemize}

Para la realización de las pruebas unitarias, se le aplicó la técnica de prueba del camino básico a las unidades
código que responden a funcionalidades críticas del software, lo cual permitió generar el grafo de flujo,
calcular la Complejidad Ciclomática (CC) para determinar los caminos linealmente independientes y el
número mínimo de escenarios de los casos de prueba para forzar la ejecución de cada camino del conjunto
básico.
Luego en apoyo a las pruebas se usó el módulo \textit{TestCase} que ofrece \textit{Django REST Framework}, para la automatización de las pruebas unitarias. Con él se probó cada módulo desarrollado, y gracias a la aplicación de la
técnica de camino básico, en aquellas funcionalidades críticas, se pudieron automatizar pruebas para cada
uno de los escenarios o caminos posibles, garantizando probar todo el código en cuestión.

Entre los elementos de código que fueron probadas se encuentra el referente al método \textit{post} de la clase
\textit{Message\_Create\_AV}, que se encarga de crear los mensajes tanto del usario como de las respuestas del sistema al usuario en una conversación asociada.

\begin{table}[h]
	\centering
	\caption{Cálculo de la complejidad ciclomática del método \textit{post} de la clase \textit{Message\_Create\_AV}}
	\label{tab:complejidad-ciclomatica}
	\begin{tabular}{|c|}
		\hline
		Método \\
		\includegraphics[width=0.4\linewidth]{images/postCreateMessage.png} \\
		\hline
		Grafo \\
		\includegraphics[width=0.4\linewidth]{images/postMessage.png} \\
		\hline
	\end{tabular}
\end{table}

\begin{table}[H]
	\centering
	\caption{Cálculo de la complejidad ciclomática del método \texttt{post} de la clase \texttt{Message\_Create\_AV}}
	\label{tab:complejidad-ciclomatica2}
	\renewcommand{\arraystretch}{1.5}
	\begin{tabular}{|>{\bfseries}m{5cm}|m{4cm}|m{4cm}|}
		\hline
		Complejidad Ciclomática: & \( V(G) = A - N + 2 \) & \( V(G) = P + 1 \) \\
		\hline
		\( V(G) = \# \textit{ de regiones} \) & \( V(G) = 9 - 9 + 2 \) & \( V(G) = 1 + 1 \) \\
		\hline
		\( V(G) = 2 \) & \( V(G) = 2 \) & \( V(G) = 2 \) \\
		\hline
	\end{tabular}
\end{table}

Luego de la determinación de los nodos y flujos de control del código se obtuvo el grafo de flujo y se calculó la complejidad ciclomática del algoritmo.
Como resultado se obtuvo que la complejidad ciclomática es igual a 2, lo que significa que existen dos posibles caminos
linealmente independientes y hay que diseñar un mínimo de dos casos de prueba para el algoritmo. La Tabla \ref{tab:caminos-grafos2} muestra los caminos existentes.

\begin{table}[h]
	\centering
	\caption{Caminos del grafo de flujo (Fuente: Elaboración propia).}
	\label{tab:caminos-grafos2}
	\begin{tabular}{|>{\bfseries}m{5cm}|m{4cm}|m{4cm}|}
		\hline
		\textbf{No.} & \textbf{Camino} \\ \hline
		1            & 1, 2, 3, 9      \\ \hline
		2            & 1, 2, 3, 4, 5, 6, 7, 8   \\ \hline
	\end{tabular}
\end{table}

Los casos de prueba para las pruebas de caja blanca por la técnica de camino básico se ejecutan por cada
camino independiente que se determine en un algoritmo específico. A continuación, se muestra el caso de
prueba para el camino básico independiente 2 del algoritmo.

\begin{longtable}{|p{4cm}|p{11cm}|}
	\caption{Caso de Prueba para el camino básico 1 (Fuente: Elaboración propia).}
	\label{tab:caminos-grafo}\\
	\hline
	\textbf{Proceso} &  \\ \hline
	\textbf{Caso de prueba} & Recibir consulta . Escenario 1.1 \\ \hline
	\textbf{Camino independiente} & 1, 2, 3, 9 \\ \hline
	\textbf{Entradas} &
	\begin{itemize}
		\item \textbf{Consulta}: Lista los barcos que entraron al puerto de la Habana en 1851.
		\item \textbf{uid\_sala\_de\_chat:} Con valor nulo.
	\end{itemize} \\ \hline
	\textbf{Resultados esperados} &
		\begin{itemize}
			\item Mensaje del sistema indicando que no existe la sala de \textit{chat}.
		\end{itemize} \\ \hline
		
	\textbf{Condiciones de ejecución} &
	\begin{itemize}
		\item El usuario debe estar autenticado.
	\end{itemize} \\ \hline
\end{longtable}

\begin{longtable}{|p{4cm}|p{11cm}|}
	\caption{Caso de Prueba para el camino básico 2 (Fuente: Elaboración propia).}
	\label{tab:caminos-grafo}\\
	\hline
	\textbf{Proceso} &  \\ \hline
	\textbf{Caso de prueba} & Recibir consulta. Escenario 1.2 \\ \hline
	\textbf{Camino independiente} & 1, 2, 3, 4, 5, 6, 7, 8 \\ \hline
	\textbf{Entradas} &
	\begin{itemize}
		\item \textbf{Consulta}: Lista los barcos que entraron al puerto de la Habana en 1851.
		\item \textbf{uid\_sala\_de\_chat:} El valor correspondiente a la conversación.
	\end{itemize} \\ \hline
	\textbf{Resultados esperados} &
	\begin{itemize}
		\item Lista de barcos que cumplan las condiciones.
		\item En la UI mostrar el mensaje generado por el sistema multiagente.
	\end{itemize} \\ \hline
	
	\textbf{Condiciones de ejecución} &
	\begin{itemize}
		\item El usuario debe estar autenticado.
		\item Debe estar una conversación creada.
	\end{itemize} \\ \hline
\end{longtable}

Con la realización de los casos de prueba diseñados se probó la ejecución de cada sentencia del código al
menos una vez, teniendo en cuenta todas las condiciones lógicas en sus variantes verdaderas y falsas. La
obtención de la complejidad ciclomática de valor 2 del método post ejemplificado, permitió determinar que existen 2 caminos
linealmente independientes, suficientes para probar el código al menos una vez.
Los resultados del método de caja blanca fueron satisfactorios, se comprobó que los caminos se ejecutaban al menos una vez para todos los casos de prueba. Se automatizaron un total de 25 casos de prueba con el uso de la biblioteca \textit{TestCase}, de los cuales a 5 se le aplicó la técnica del camino básico, permitiendo que su automatización garantice probar todos los caminos con un mínimo de escenarios diseñados,
y obteniendo 0 errores como se aprecia en la Figura \ref{fig:unit_tests}.

\begin{figure}[htbp] % h: here, t: top, b: bottom, p: page of floats - ajusta según necesidad
	\centering
	% Asegúrate de que la ruta 'images/arquitectura_web.png' sea correcta
	\includegraphics[width=0.9\textwidth]{images/TestCase.PNG} 
	\caption{Resultado de las pruebas unitarias.}
	\label{fig:unit_tests}
\end{figure}

\section{Pruebas funcionales}

Este tipo de prueba se realiza sobre el sistema funcionando, comprobando que cumpla con la especificación. Para estas pruebas, se utilizan las especificaciones de casos de prueba. Las pruebas basadas en requerimientos son pruebas de validación más que de defecto: se intenta demostrar que el sistema implementó adecuadamente sus requerimientos~\cite{sommerville2011software}.

\subsection{Método de caja negra}
Las pruebas de caja negra, también llamadas pruebas de comportamiento, se enfocan en los requerimientos funcionales del software. Las técnicas de prueba de caja negra permiten derivar conjuntos de condiciones de entrada que revisarán los requerimientos funcionales para un programa~\cite{pressman2010practitioner}. El método de caja negra presenta varias técnicas de prueba como son: partición de equivalencia y análisis de valores límites.
En la presente investigación se utilizará específicamente dentro del método de caja negra la técnica de partición de equivalencia generando los casos de pruebas de dicha técnica
sobre las diferentes interfaces que responden a los requisitos funcionales. Para la aplicación de pruebas de regresión sobre los casos de prueba definidos se usará la herramienta \textit{Selenium IDE} (Figura \ref{fig:unit_test}), que permite grabar todas las interacciones de un usuario con el navegador y posibilita ejecutar de forma automática las mismas, reduciendo el tiempo y los costos de las pruebas funcionales.

\begin{figure}[htbp] % h: here, t: top, b: bottom, p: page of floats - ajusta según necesidad
	\centering
	% Asegúrate de que la ruta 'images/arquitectura_web.png' sea correcta
	\includegraphics[width=0.42\textwidth]{images/Pruebas_funcionales.PNG} 
	\caption{Representación del resultado la ejecución de una prueba usando Selenium IDE, del requisito Insertar consulta.}
	\label{fig:unit_test}
\end{figure}

A continuación, la Tabla \ref{tab:caso_prueba_enviar_consulta}  muestra el diseño de caso de pruebas del requisito “Insertar consulta” donde se analizarán las variables y condiciones que puedan determinar la respuesta del sistema.


\begin{small} 
	\begin{longtable}{|p{2.2cm}|p{3cm}|p{3.2cm}|p{3.2cm}|p{3.2cm}|}
		\caption{Caso de prueba para la funcionalidad Insertar Consulta (Fuente: Elaboración Propia).} \label{tab:caso_prueba_enviar_consulta} \\
		\hline
		\textbf{Escenario} & \textbf{Descripción} & \textbf{Variables (Mensaje)} & \textbf{Respuesta Esperada} & \textbf{Respuesta} \\
		\hline
		\endfirsthead
		
		\hline
		\textbf{Escenario} & \textbf{Descripción} & \textbf{Variables (Consulta)} & \textbf{Respuesta Esperada} & \textbf{Respuesta} \\
		\hline
		\endhead
		
		\hline
		\endfoot
		
		\hline
		\endlastfoot
		
		EC 1.1. Enviar consulta correctamente. & El usuario debe escribir la consulta y dar clic en el botón de enviar. & ¿Cuantos barcos entraron al puerto de La Habana en 1851? & 567 & 567 \\
		\hline
		EC 1.2. Enviar consulta incorrecta(con un contexto fuera de la información conocida por el sistema). & El usuario debe escribir la consulta y dar clic en el botón de enviar. & ¿Cuánto tiempo duró la 2da guerra mundial? & Lo lamento no tengo esa información disponible. & La información solicitada no se encuentra. Por favor reformule la consulta. \\
		\hline
		EC 1.3. Solicitar un consulta cuya respuesta sea una gráfica. & El usuario debe escribir la consulta y dar clic en el botón de enviar. & Genera una gráfica con el \% de los barcos que entraron a La Habana con arroz en diferentes años. & La respuesta debe ser una gráfica. & Responde con una gráfica correspondiente acorde a la consulta. \\
		\hline
		\hline
		EC 1.4. Enviar consulta vacía & El usuario debe dar clic en el botón de enviar. & - & No se efectúa el envió de consultas vacías. & No se envía la consulta vacía. \\
		\hline
		
	\end{longtable}
\end{small}

\begin{table}[H]
	\caption{Variables de caso de prueba “Insertar Consulta” (Fuente: Elaboración Propia).}
	\label{tab:variables_insertar_consulta}
	\centering
	\begin{tabular}{|c|c|c|p{8.5cm}|}
		\hline
		\textbf{No.} & \textbf{Variable} & \textbf{Valor Nulo} & \textbf{Descripción} \\
		\hline
		1 & Consulta & No & Es un campo de texto que permite al usuario escribir una consulta al sistema \\
		\hline
	\end{tabular}
	
\end{table}

Las pruebas de caja negra se aplicaron con el objetivo de evaluar las interfaces de comunicación con el
usuario, las que demostraron coherencia y funcionalidad, así como probar todas aquellas funcionalidades
directamente relacionadas con los requisitos funcionales del sistema. La técnica de partición de equivalencia
es aplicada para evaluar los diferentes escenarios que pueden tener lugar ante la ejecución de una acción.
Como resultado de la aplicación de estas pruebas se ejecutan las posibles variantes que posee una interfaz
de comunicación con el usuario, resolviendo las no conformidades arrojadas y perfeccionando lo obtenido.

En este proceso se evaluaron las distintas variantes de la interfaz de comunicación con el usuario, permitiendo identificar y corregir no conformidades, así como perfeccionar la solución propuesta. Durante el proceso de pruebas se ejecutaron un total de 8 casos de prueba y tres iteraciones de prueba. En la primera iteración se identificaron 5 no conformidades, clasificadas en dos categorías: de funcionalidad y validación de datos. En la segunda iteración, mediante pruebas de regresión automatizadas con \textit{Selenium IDE}, se comprobó la corrección de las no conformidades detectados previamente, aunque surgieron 2 nuevas no conformidades relacionadas únicamente con la validación. Finalmente, en la tercera iteración no se detectaron nuevas no conformidades, cumpliéndose satisfactoriamente los requisitos funcionales definidos. La Figura \ref{fig:grafica_rf} ilustra la evolución de los resultados obtenidos en cada iteración.
\begin{figure}[htbp] % h: here, t: top, b: bottom, p: page of floats - ajusta según necesidad
	\centering
	% Asegúrate de que la ruta 'images/arquitectura_web.png' sea correcta
	\includegraphics[width=0.5\textwidth]{images/grafica_pruebas_funcionales.PNG} 
	\caption{Representación del resultado de las pruebas funcionales (Fuente: Elaboración Propia).}
	\label{fig:grafica_rf}
\end{figure}



\section{Pruebas de rendimiento}

Las pruebas de rendimiento deben diseñarse para garantizar que el sistema procese su carga pretendida. Esto
implica efectuar una serie de pruebas donde se aumenta la carga, hasta que el rendimiento del sistema se
vuelve inaceptable. Las pruebas de rendimiento se preocupan tanto por demostrar que el sistema cumple
con sus requerimientos, como por descubrir problemas y defectos en el sistema~\cite{sommerville2011software}.

Para la realización de las pruebas de rendimiento del sistema se utilizó la herramienta Locust destinada para
la ejecución de estas pruebas mediante código python. La cuál permitió probar la aplicación simulando un
entorno similar al de producción, donde actuaban de forma concurrente 20 usuarios, realizando alrededor de 5 peticiones por segundo, obteniendo un tiempo de respuesta máximo
menor que cinco segundos y en el punto final más critico del sistema se obtuvo para el 95\% de los casos un tiempo de respuesta promedio de 11 segundos, cumpliendo así con lo pactado con el cliente en los requisitos no funcionales del
sistema. Estos resultados se pueden apreciar en la Figura \ref{fig:rend1}. Es fundamental interpretar los resultados de las pruebas de rendimiento presentados anteriormente teniendo en cuenta las especificaciones del entorno de prueba y las características de los servicios externos involucrados, como el proveedor de Inteligencia Artificial.

Las pruebas de rendimiento se ejecutaron a la API Django operando en una máquina con las siguientes características:

\begin{itemize}
	\item \textbf{Procesador:} Intel(R) Core(TM) i7-2670QM CPU @ 2.20GHz
	\item \textbf{Velocidad Base del Procesador:} 2.20 GHz (2201 MHz)
	\item \textbf{Núcleos Físicos:} 4 procesadores principales
	\item \textbf{Procesadores Lógicos (Hilos):} 8 procesadores lógicos
\end{itemize}

Para la generación de respuestas por IA, la aplicación se integró con el servicio Google Gemini a través de su API. Esto implica que cada vez que se requiere una respuesta de la IA se envía una solicitud a la API de Gemini~\footnote{La API de Google Gemini es una interfaz desarrollada por Google que permite integrar en tus aplicaciones los modelos de inteligencia artificial generativa más avanzados de Google, conocidos como Gemini. Estos modelos son nativos multimodales: pueden procesar y generar respuestas a partir de texto, imágenes, audio, video y documentos no estructurados como PDFs.} que tiene tiempos de respuestas de mayor velocidad que si se utiliza un modelo en local. Los resultados actuales deben considerarse como una línea base específica para este entorno y esta configuración de IA. Es altamente probable que las métricas de rendimiento cambien (y potencialmente mejoren significativamente) bajo otras circunstancias donde se cuente con mayor cantidad de RAM~\footnote{La RAM (Memoria de Acceso Aleatorio, o Random Access Memory en inglés) es un tipo de memoria principal que utiliza tu computadora, teléfono móvil u otros dispositivos para almacenar temporalmente los datos y programas que se están utilizando en ese momento.}, mejor procesador y almacenamiento SSD~\footnote{El almacenamiento SSD (\textit{Solid State Drive} o unidad de estado sólido) es un dispositivo de almacenamiento de datos que utiliza memoria flash para almacenar información, a diferencia de los discos duros tradicionales (HDD), que usan discos magnéticos y partes móviles.}.\\

\begin{figure}[htbp] % h: here, t: top, b: bottom, p: page of floats - ajusta según necesidad
	\centering
	% Asegúrate de que la ruta 'images/arquitectura_web.png' sea correcta
	\includegraphics[width=0.9\textwidth]{images/Rendimiento1.PNG} 
	\caption{ Prueba de rendimiento en Locust}
	\label{fig:rend1}
\end{figure}

\section{Pruebas de seguridad}

Las pruebas de seguridad se diseñan para sondear las vulnerabilidades del entorno del lado del cliente, las
comunicaciones de red que ocurren conforme los datos pasan de cliente a servidor y viceversa, y el entorno
del lado servidor. Cada uno de estos dominios puede atacarse, y es tarea del examinador de seguridad
descubrir las debilidades que puedan explotar quienes tengan intención de hacerlo~\cite{pressman2010practitioner}.

Las pruebas de seguridad se aplicaron con ayuda de la herramienta Acunetix Web Vulnerability Scanner 9.5
que establece alertas de tipo: alta, media, baja e informacional, realizándose en dos iteraciones durante el
desarrollo de la propuesta solución.
En una primera iteración se obtuvo un total de 22 alertas de seguridad, de las cuales 4 clasifican de nivel
medio, 1 de nivel bajo y 17 informativas.

De las de nivel medio, se destacaron el uso de protocolo no seguro
para el envío de datos, así como los mensajes de error que se muestra en el modo DEBUG de Django para
el desarrollo y se detectaron problemas para la protección de contra ataques de fuerza bruta en el formulario
de autenticación.\\
La de nivel bajo, consistía en vistas del sitio que se podían acceder directamente sin pasar la autenticación
y el sistema de roles establecido. De carácter informativo fueron detectadas posibles cuentas de usuario en ficheros, así como presencia de directorios desprotegidos y la existencia de etiquetas iframe de HTML5.

Después de aplicar refactorización del código y realizar las validaciones correspondientes, se aplicó la segunda iteración en búsqueda de vulnerabilidades al sistema, arrojando como resultado que todas las que se habían detectado en la primera iteración, habían sido corregidas.

Resultados de las iteraciones: 

\begin{figure}[htbp] % h: here, t: top, b: bottom, p: page of floats - ajusta según necesidad
	\centering
	% Asegúrate de que la ruta 'images/arquitectura_web.png' sea correcta
	\includegraphics[width=0.5\textwidth]{images/primeraIt.PNG} 
	\caption{ Prueba de seguridad 1ra iteración.}
	\label{fig:grafica_segur}
\end{figure}

\begin{figure}[htbp] % h: here, t: top, b: bottom, p: page of floats - ajusta según necesidad
	\centering
	% Asegúrate de que la ruta 'images/arquitectura_web.png' sea correcta
	\includegraphics[width=0.5\textwidth]{images/2iter.PNG} 
	\caption{ Prueba de seguridad 2da iteración.}
	\label{fig:grafica_segu2}
\end{figure}

%\section{Pruebas de aceptación}
%\label{sec:pruebas-aceptacion}

%Las pruebas de aceptación, también conocidas como pruebas de usuario, constituyen la fase final de verificación antes de la entrega o despliegue del software. Su objetivo principal es validar que el sistema cumple con las necesidades y expectativas del cliente o usuario final, y que es apto para su propósito en el entorno operativo real o simulado~\cite{pressman2010practitioner}. Estas pruebas se realizan desde la perspectiva del usuario, enfocándose en la funcionalidad global y la usabilidad del sistema, en lugar de en los detalles técnicos internos.

%Para el sistema multiagente conversacional desarrollado, se planificaron \textbf{pruebas alfa} como principal método de prueba de aceptación. Este tipo de prueba implica la participación de un grupo selecto de usuarios finales o representantes del cliente (en este caso, podrían ser historiadores, investigadores o académicos interesados en el \textit{Diario de la Marina}), quienes interactúan con el sistema en un entorno controlado, usualmente el entorno de desarrollo o uno de staging muy similar al de producción.

%\textbf{Metodología de las Pruebas Alfa:}
%\begin{itemize}
	%\item \textbf{Selección de Participantes:} Se identificarán usuarios con conocimiento del dominio histórico del \textit{Diario de la Marina} y con diferentes niveles de familiaridad con herramientas tecnológicas.
	%\item \textbf{Definición de Escenarios de Uso:} Se prepararán escenarios de prueba basados en casos de uso reales y representativos, como la formulación de consultas históricas específicas, la solicitud de contextualización de eventos, y la interacción general con el asistente conversacional para explorar la información contenida en el periódico.
	%\item \textbf{Ejecución de Pruebas:} Los participantes utilizarán el sistema para completar los escenarios definidos. Se les alentará a explorar libremente y a registrar cualquier problema, duda, sugerencia o aspecto que consideren relevante.
	%\item \textbf{Recopilación de Feedback:} Se utilizarán cuestionarios, entrevistas y observación directa para recopilar la retroalimentación de los participantes. Se prestará especial atención a:
	%\begin{itemize}
		%\item Facilidad de uso de la interfaz.
		%\item Claridad y precisión de las respuestas generadas por el sistema.
		%\item Relevancia y contextualización de la información recuperada.
		%\item Satisfacción general con la interacción.
		%\item Identificación de posibles errores funcionales o comportamientos inesperados no detectados en fases anteriores.
	%\end{itemize}
	%\item \textbf{Análisis de Resultados:} La retroalimentación recopilada será analizada para identificar patrones, problemas recurrentes y áreas de mejora. Estos hallazgos servirán para realizar ajustes finales al sistema antes de su posible despliegue o para planificar futuras iteraciones de desarrollo.
%\end{itemize}

%El objetivo de estas pruebas de aceptación no es solo confirmar que el sistema "funciona", sino asegurar que realmente aporta valor al usuario final y que su integración en un flujo de trabajo de investigación histórica sería beneficiosa y eficiente. La validación por parte de los usuarios clave es crucial para el éxito y la adopción del sistema multiagente conversacional.

%Como resultado de haber aplicado las pruebas alfa se identificaron tres nuevas no conformidades las cuales fueron resueltas en su totalidad.

% (Opcional: Si ya se realizaron algunas pruebas alfa y tienes resultados preliminares, podrías añadir un breve párrafo aquí. Si no, puedes dejarlo como la planificación).
% Ejemplo si tienes resultados:
% "En una fase preliminar de pruebas alfa con un grupo reducido de X investigadores, se observó una recepción generalmente positiva hacia la capacidad del sistema para [mencionar un logro clave]. Sin embargo, se identificaron áreas de mejora en [mencionar un área], las cuales serán abordadas en futuras iteraciones."

%\subsection{Satisfacción de los usuarios}

%\begin{table}[htbp]
%	\caption{Cuadro Lógico de Iadov para la Evaluación del Sistema Multiagente Conversacional (Fuente: Elaboración propia).}
%	\label{tab:cuadro-iadov-foto} % Nueva etiqueta para esta tabla de imagen
%	\centering
%	\includegraphics[width=0.8\textwidth]{images/Tabla de iadov.PNG}\\
%\end{table}

%Con el objetivo de evaluar el sistema implementado se utiliza la técnica de Iadov, esta técnica evalúa el nivel de satisfacción del usuario, permitiendo conocer si la solución propuesta cumple con las expectativas esperadas. Esta técnica constituye una vía indirecta para el estudio de la satisfacción, ya que los criterios que se utilizan se fundamentan en las relaciones que se establecen entre tres preguntas cerradas (preguntas 1, 2 y 3) que se intercalan dentro de un cuestionario (Ver \textbf{Anexo C})~\cite{tinajero2021tecnica}. Estas tres preguntas se relacionan a través de lo que se denomina el “Cuadro Lógico de Iadov”, el cual se muestra a continuación en la Tabla \ref{tab:cuadro-iadov-foto}.

%El número resultante de la interrelación de las tres preguntas indica la posición de cada sujeto en la escala de satisfacción individual y grupal.


\section*{Conclusiones del capítulo}
\addcontentsline{toc}{section}{Conclusiones del Capítulo}
\label{sec:conclusiones-cap3}

La fase de pruebas del sistema multiagente conversacional ha proporcionado una validación integral de su funcionalidad y rendimiento, arrojando resultados significativos sobre su estado actual y áreas de potencial optimización como son la latencia de respuesta y la mejora de su contextualización.

Los componentes individuales del sistema demostraron una alta fiabilidad a nivel unitario, superando con éxito un total de 25 casos de prueba automatizados, incluyendo aquellos diseñados con la técnica del camino básico para requisitos funcionales críticos, sin registrarse errores tras las correcciones pertinentes. Esto subraya la robustez interna del código desarrollado.

Desde la perspectiva funcional, el sistema cumplió con el 100\% de los requisitos especificados. Las pruebas de caja negra, iteradas hasta la corrección de todas las no conformidades identificadas, confirmaron que las interacciones del usuario y las respuestas del sistema se comportan según lo esperado, validando la correcta implementación de las funcionalidades de cara al usuario.

Las pruebas de rendimiento, ejecutadas con \textit{Locust} bajo una carga de 20 usuarios concurrentes, indicaron que la mayoría de los \textit{endpoints} de la \textit{API} operan con tiempos de respuesta eficientes (inferiores a 5 segundos), el \textit{endpoint} de envío de mensajes y generación de respuesta por IA (\textit{`POST /api/chats/[uid]/messages/`}) presenta una latencia considerable, con un tiempo de respuesta promedio para el 95\% de los casos de 11 segundos debido a su alta complejidad y dependencia del microservicio del sistema multiagente.

En el ámbito de la seguridad, el análisis mediante \textit{Acunetix Web Vulnerability Scanner} reveló y permitió subsanar vulnerabilidades de nivel medio y bajo en una primera iteración, incluyendo la protección contra ataques comunes y la corrección de configuraciones inseguras. La segunda iteración de pruebas confirmó la efectividad de estas mitigaciones, resultando en un sistema sin alertas de seguridad críticas.  

En síntesis, las pruebas realizadas confirman que el sistema multiagente es funcionalmente completo, seguro contra las vulnerabilidades evaluadas y cumple con los umbrales de rendimiento establecidos, al tiempo que señalan la interacción con la IA como un área prioritaria para futuras optimizaciones enfocadas en la escalabilidad y la mejora de la experiencia de usuario en tiempo real. % Capítulo 3
	\conclusions
Como resultados de la investigaci�n y dando cumplimiento al objetivo general, se arrib� a las siguientes conclusiones:

\begin{itemize}
	\item Conclusi�n 1
	
	\item Conclusi�n 2 ...
	
	\item Conclusi�n n
\end{itemize} % Conclusiones
	\suggestions
\begin{itemize}
	\item Recomendaci�n 1
	
	\item Recomendaci�n 2 ...
	
	\item Recomendaci�n n
\end{itemize} % Recomendaciones
	
	
	% ============================
	% ANEXOS Y REFERENCIAS
	% ============================
	\appendix
	\appendixes

\renewcommand{\appendixname}{Anexo}
\renewcommand{\appendixtocname}{Anexos}
\renewcommand{\appendixpagename}{Anexos}
%\newcounter{anexo1}
%\setcounter{anexo1}{1}
\begin{addendum}
	
	\chapter{Historias de Usuario}
	% ==========================================
	% Gestión de Usuarios y Autenticación
	% ==========================================
	
	\begin{userstory}[hu:01]
		\storyname{Registrar una nueva cuenta de usuario}
		\storyuser{Visitante del sitio web}
		\storyiter{1} % Iteración estimada
		\storypriority{Alta} % Basado en RF1
		\storyrisk{Bajo}
		\storypoints{1 semana} % Estimación basada en ejemplo
		\storyprogrammer{Daniel Rojas Grass} % Manteniendo el nombre del ejemplo
		\storydescription{
			Como visitante del sitio web, debe poder registrar una nueva cuenta proporcionando su dirección de correo electrónico, nombre de usuario y una contraseña segura, para crear una cuenta personal que la permita acceder a las funcionalidades de consulta y visualización del archivo histórico y guardar su historial de conversaciones. (Corresponde principalmente a RF1)
			
			\textbf{Precondiciones:}
			\begin{itemize}
				\item El visitante no tiene una sesión activa.
				\item El visitante se encuentra en la página o sección de registro del sitio web.
				\item El \textit{backend} está operativo y accesible.
			\end{itemize}
			
			\textbf{Flujo de acción:}
			\begin{enumerate}
				\item Visitante ingresa dirección de correo electrónico, nombre de usuario, contraseña y confirmación de contraseña en el formulario de la sección registrar cuenta.
				\item Visitante envía el formulario.
				\item El sitio web valida formato básico (e.g., dirección de correo electrónico válido, contraseñas coinciden, nombre de usuario único).
				\item El sitio web envía los datos al endpoint de registro del \textit{backend}.
				\item El \textit{backend} valida los datos (e.g., dirección de correo electrónico no existente,nombre de usuario no existente, complejidad de contraseña) y crea el usuario en la base de datos de forma segura.
				\item El \textit{backend} retorna una respuesta de éxito o error al frontend.
				\item El \textit{frontend} muestra un mensaje apropiado al usuario (éxito o error específico).
			\end{enumerate}
		}
		\storyobservation{
			Implementar validación de fortaleza de contraseña en \textit{frontend} y \textit{backend}. Asegurar almacenamiento seguro de contraseñas (hashing). Los mensajes de error deben ser claros (e.g., 'El correo electrónico ya está registrado', "Las contraseñas no coinciden").
		}
		\storyinterface{ % Inicio del contenido de storyinterface
			Formulario de registro del sitio web: % Texto descriptivo inicial
			\par\medskip % Añade un pequeño espacio vertical
			\begin{center} % Para centrar la imagen
				\includegraphics[width=0.6\textwidth]{images/create.PNG} % Imagen del ejemplo
			\end{center}
			\medskip % Espacio después de la imagen
		}
		
		
	\end{userstory}
	
	\begin{userstory}[hu:02]
		\storyname{Iniciar sesión en el sistema}
		\storyuser{Usuario registrado}
		\storyiter{1} % Iteración estimada
		\storypriority{Alta} % Basado en RF2
		\storyrisk{Bajo}
		\storypoints{1 semana} % Estimación basada en ejemplo
		\storyprogrammer{Daniel Rojas Grass}
		\storydescription{
			Como usuario registrado, debe poder iniciar sesión utilizando su correo electrónico y contraseña previamente registrados, mi historial de conversaciones y utilizar las capacidades completas del sistema de chat. (Corresponde principalmente a RF2, RF4)
			
			\textbf{Precondiciones:}
			\begin{itemize}
				\item El usuario tiene una cuenta previamente registrada en el sistema.
				\item El usuario no tiene una sesión activa.
				\item El usuario se encuentra en la página o sección de inicio de sesión del sitio web.
				\item El \textit{backend} está operativo y accesible.
			\end{itemize}
			
			\textbf{Flujo de acción:}
			\begin{enumerate}
				\item Usuario ingresa su dirección de correo electrónico y contraseña en el formulario del sitio web.
				\item Usuario envía el formulario.
				\item El sitio web envía las credenciales al \textit{endpoint} de autenticación del \textit{backend}.
				\item El \textit{backend} verifica las credenciales contra la base de datos.
				\item Si las credenciales son válidas, el \textit{backend} genera un token de sesión (e.g., JWT) y lo retorna al \textit{frontend}.
				\item Si las credenciales son inválidas, el backend retorna un error de autenticación.
				\item El \textit{frontend} almacena el token de sesión de forma segura (e.g., localStorage, sessionStorage o cookie HttpOnly).
				\item Si el inicio de sesión es exitoso, el \textit{frontend} redirige al usuario a la interfaz principal del chat. Si falla, muestra un mensaje de error ("Credenciales incorrectas").
			\end{enumerate}
		}
		\storyobservation{
			Utilizar HTTPS para la comunicación. Implementar medidas contra ataques de fuerza bruta (e.g., límites de intentos). El manejo del token en el \textit{frontend} debe seguir las mejores prácticas de seguridad.
		}
		\storyinterface{
			Formulario de inicio de sesión del sitio web:
			\par\medskip % Añade un pequeño espacio vertical
			\begin{center} % Para centrar la imagen
				\includegraphics[width=0.6\textwidth]{images/loguin.PNG} % Imagen del ejemplo
			\end{center}
			\medskip
		}
		
	\end{userstory}
	
	\begin{userstory}[hu:03]
		\storyname{Cerrar sesión del sistema}
		\storyuser{Usuario autenticado}
		\storyiter{1} % Iteración estimada
		\storypriority{Media} % Basado en RF3
		\storyrisk{Bajo}
		\storypoints{0.5 semanas} % Estimación basada en ejemplo
		\storyprogrammer{Daniel Rojas Grass}
		\storydescription{
			Como usuario autenticado, debe disponer de una opción clara para cerrar su sesión activa, para asegurar la privacidad de su cuenta y finalizar su interacción con el sistema de forma segura. (Corresponde principalmente a RF3)
			
			\textbf{Precondiciones:}
			\begin{itemize}
				\item El usuario tiene una sesión activa (posee un token válido).
				\item El usuario está en la interfaz principal del sistema.
			\end{itemize}
			
			\textbf{Flujo de acción:}
			\begin{enumerate}
				\item Usuario hace clic en el botón o enlace 'Cerrar Sesión'.
				\item El \textit{frontend} elimina el token de sesión almacenado localmente.
				\item (Recomendado) El \textit{frontend} envía una solicitud al \textit{backend} para invalidar el token en el servidor (si se usa una blacklist de tokens).
				\item El \textit{frontend} redirige al usuario a la página de inicio de sesión o a una página pública principal.
				\item Cualquier intento posterior de acceder a rutas protegidas con el token antiguo debe fallar.
			\end{enumerate}
		}
		\storyobservation{
			El botón de cerrar sesión debe ser fácilmente accesible en la interfaz de usuario autenticado.
		}
		\storyinterface{Botón cerrar sesión en el sitio web:
			\par\medskip % Añade un pequeño espacio vertical
			\begin{center} % Para centrar la imagen
				\includegraphics[width=0.6\textwidth]{images/cerrar.PNG} % Imagen del ejemplo
			\end{center}
		}
		
	\end{userstory}
	
	% ==========================================
	% Gestión de Conversaciones
	% ==========================================
	
	\begin{userstory}[hu:04]
		\storyname{Iniciar una nueva conversación}
		\storyuser{Usuario autenticado}
		\storyiter{2} % Iteración estimada
		\storypriority{Alta} % Basado en RF5
		\storyrisk{Bajo}
		\storypoints{0.5 semanas} % Estimación basada en ejemplo
		\storyprogrammer{Daniel Rojas Grass}
		\storydescription{
			Como usuario autenticado, debe poder iniciar una nueva conversación de chat en cualquier momento, para realizar consultas sobre temas distintos sin mezclar las interacciones o para empezar una nueva conversación si lo necesita. (Corresponde principalmente a RF5)
			
			\textbf{Precondiciones:}
			\begin{itemize}
				\item El usuario tiene una sesión activa.
				\item El usuario se encuentra en la interfaz principal del chat.
			\end{itemize}
			
			\textbf{Flujo de acción:}
			\begin{enumerate}
				\item Usuario hace clic en la opción "Nueva Conversación" (o icono '+').
				\item El \textit{frontend} limpia el área de visualización del chat actual.
				\item El \textit{frontend} establece un estado interno que indica que la próxima consulta pertenece a una nueva conversación (puede no requerir llamada inmediata al \textit{backend}).
				\item Opcionalmente, el \textit{frontend} puede asignar un ID temporal a la nueva conversación hasta que se envíe el primer mensaje.
				\item La interfaz se muestra lista para recibir la primera consulta de la nueva conversación.
			\end{enumerate}
		}
		\storyobservation{
			La acción debe ser visualmente clara y distinguible de seleccionar una conversación existente. El estado de "nueva conversación" debe manejarse correctamente hasta el primer envío.
		}
		\storyinterface{Botón nueva conversación en el sitio web:
			\par\medskip % Añade un pequeño espacio vertical
			\begin{center} % Para centrar la imagen
				\includegraphics[width=0.6\textwidth]{images/newConversation.PNG} % Imagen del ejemplo
			\end{center}
			\medskip
		}
		
	\end{userstory}
	
	\begin{userstory}[hu:05]
		\storyname{Ver historial de conversaciones}
		\storyuser{Usuario autenticado}
		\storyiter{2} % Iteración estimada
		\storypriority{Media} % Basado en RF6
		\storyrisk{Bajo}
		\storypoints{1 semana} % Estimación basada en ejemplo
		\storyprogrammer{Daniel Rojas Grass}
		\storydescription{
			Como usuario autenticado, debe ver una lista organizada de sus conversaciones anteriores (por ejemplo, con un título autogenerado o fecha), para poder identificar y acceder fácilmente a interacciones pasadas. (Corresponde principalmente a RF6)
			
			\textbf{Precondiciones:}
			\begin{itemize}
				\item El usuario tiene una sesión activa.
				\item El \textit{backend} y la base de datos que almacena el historial están operativos.
			\end{itemize}
			
			\textbf{Flujo de acción:}
			\begin{enumerate}
				\item El \textit{frontend} (al cargar la interfaz principal o al interactuar con un panel de historial) solicita la lista de conversaciones del usuario al \textit{backend}.
				\item El backend consulta la base de datos para obtener los metadatos de las conversaciones asociadas al usuario autenticado (ID, título/fecha, última actualización).
				\item El \textit{backend} retorna la lista de conversaciones al \textit{frontend}.
				\item El \textit{frontend} muestra la lista en un panel lateral o sección designada, permitiendo al usuario ver los identificadores de cada conversación.
			\end{enumerate}
		}
		\storyobservation{
			Considerar paginación si el historial puede ser muy largo. La generación de títulos/identificadores debe ser útil (e.g., basado en la primera consulta). La lista debe actualizarse si se crea o elimina una conversación.
		}
		\storyinterface{Panel/Sección de Historial en el sitio web:
			\par\medskip % Añade un pequeño espacio vertical
			\begin{center} % Para centrar la imagen
				\includegraphics[width=0.4\textwidth]{images/histo.PNG} % Imagen del ejemplo
			\end{center}
			\medskip
		}
		
	\end{userstory}
	
	\begin{userstory}[hu:06]
		\storyname{Abrir una conversación del historial}
		\storyuser{Usuario autenticado}
		\storyiter{2} % Iteración estimada
		\storypriority{Media} % Basado en RF7
		\storyrisk{Bajo}
		\storypoints{1 semana} % Estimación basada en ejemplo
		\storyprogrammer{Daniel Rojas Grass}
		\storydescription{
			Como usuario autenticado, debe poder seleccionar una conversación específica de su historial listado, para cargar su contenido completo (consultas y respuestas) en la interfaz principal del chat y, opcionalmente, continuarla. (Corresponde principalmente a RF7)
			
			\textbf{Precondiciones:}
			\begin{itemize}
				\item El usuario tiene una sesión activa.
				\item El usuario está viendo la lista de su historial de conversaciones (HU:05).
				\item El \textit{backend} y la base de datos que almacena el historial están operativos.
			\end{itemize}
			
			\textbf{Flujo de acción:}
			\begin{enumerate}
				\item Usuario hace clic en una conversación específica en la lista del historial.
				\item El \textit{frontend} envía una solicitud al \textit{backend} pidiendo el contenido completo de la conversación seleccionada (pasando su ID).
				\item El \textit{backend} recupera todas las consultas y respuestas asociadas a esa conversación para ese usuario.
				\item El \textit{backend} retorna el historial completo de mensajes de esa conversación al \textit{frontend}.
				\item El \textit{backend} limpia el área de chat actual y muestra los mensajes recuperados en el orden correcto.
				\item El \textit{backend} establece la conversación seleccionada como la "conversación activa" actual.
				\item La interfaz permite al usuario añadir nuevas consultas a esta conversación activa.
			\end{enumerate}
		}
		\storyobservation{
			La carga debe ser eficiente, especialmente para conversaciones largas. La interfaz debe indicar claramente cuál conversación del historial está activa.
		}
		\storyinterface{Interacción con lista de historial y chat principal en el sitio web:
			\par\medskip % Añade un pequeño espacio vertical
			\begin{center} % Para centrar la imagen
				\includegraphics[width=0.6\textwidth]{images/lista.PNG} % Imagen del ejemplo
			\end{center}
			\medskip
		}
		
	\end{userstory}
	
	\begin{userstory}[hu:07]
		\storyname{Eliminar una conversación del historial}
		\storyuser{Usuario autenticado}
		\storyiter{3} % Iteración estimada
		\storypriority{Baja} % Basado en RF8
		\storyrisk{Bajo} % Riesgo de pérdida de datos si no hay confirmación
		\storypoints{1 semana} % Estimación basada en ejemplo
		\storyprogrammer{Daniel Rojas Grass}
		\storydescription{
			Como usuario autenticado, debe tener la opción de eliminar permanentemente una conversación específica de su historial, para mantener su historial limpio y relevante. (Corresponde principalmente a RF8)
			
			\textbf{Precondiciones:}
			\begin{itemize}
				\item El usuario tiene una sesión activa.
				\item El usuario está viendo la lista de su historial o tiene una conversación cargada que desea eliminar.
				\item El \textit{backend} y la base de datos que almacena el historial están operativos.
			\end{itemize}
			
			\textbf{Flujo de acción:}
			\begin{enumerate}
				\item Usuario hace clic en la opción "Eliminar" asociada a una conversación en la lista del historial (o en la conversación activa).
				\item El \textit{frontend} muestra un diálogo de confirmación ("¿Estás seguro de que quieres eliminar esta conversación? Esta acción no se puede deshacer.").
				\item Si el usuario confirma la eliminación:
				\begin{enumerate}
					\item El frontend envía una solicitud al backend DRF para eliminar la conversación (pasando su ID).
					\item El backend verifica que la conversación pertenece al usuario y la elimina de la base de datos.
					\item El backend retorna una respuesta de éxito al frontend.
					\item El frontend elimina la conversación de la lista visible en el historial.
					\item Si la conversación eliminada era la activa, el frontend limpia el área de chat o carga una conversación por defecto/nueva.
				\end{enumerate}
				\item Si el usuario cancela, no se realiza ninguna acción.
			\end{enumerate}
		}
		\storyobservation{
			La confirmación es crucial para prevenir eliminaciones accidentales. La eliminación debe ser lógicamente completa en el backend (borrado permanente).
		}
		\storyinterface{Opción de Eliminar en la lista de Historial o Chat activo:
			\par\medskip % Añade un pequeño espacio vertical
			\begin{center} % Para centrar la imagen
				\includegraphics[width=0.6\textwidth]{images/eliminarC.PNG} % Imagen del ejemplo (asumiendo que muestra un icono/botón de eliminar)
			\end{center}
			\medskip
		}
		
	\end{userstory}
	
	
	\chapter{Targetas CRC}
		
		\begin{longtable}{|l|l|}
			\caption{Tarjeta CRC: Usuario} \label{tablacrc6} \\
			\hline
			\multicolumn{2}{|c|}{\textbf{Tarjeta CRC}} \\
			\hline
			\textbf{Clase} & \textbf{Usuario} \\
			\hline
			\parbox[t]{0.45\linewidth}{\textbf{Responsabilidades:} \\ 
				Proporcionar datos para registro (email, username, contraseña) \\ 
				Almacenar credenciales de forma segura \\ 
				Mantener información de sesión activa \\ 
				Asociar conversaciones al usuario} 
			& 
			\parbox[t]{0.45\linewidth}{\textbf{Colaboración:} \\ 
				Autenticador \\ 
				BaseDeDatos \\ 
				Conversación} \\
			\hline
		\end{longtable}
		
		\begin{longtable}{|l|l|}
			\caption{Tarjeta CRC: Autenticador} \label{tablacrc7} \\
			\hline
			\multicolumn{2}{|c|}{\textbf{Tarjeta CRC}} \\
			\hline
			\textbf{Clase} & \textbf{Autenticador} \\
			\hline
			\parbox[t]{0.45\linewidth}{\textbf{Responsabilidades:} \\ 
				Validar datos de registro (email único, contraseña fuerte) \\ 
				Autenticar credenciales de inicio de sesión \\ 
				Generar y gestionar tokens de sesión \\ 
				Finalizar sesiones activas} 
			& 
			\parbox[t]{0.45\linewidth}{\textbf{Colaboración:} \\ 
				Usuario \\ 
				BaseDeDatos \\ 
				Backend} \\
			\hline
		\end{longtable}
		
		\begin{longtable}{|l|l|}
			\caption{Tarjeta CRC: Conversación} \label{tablacrc8} \\
			\hline
			\multicolumn{2}{|c|}{\textbf{Tarjeta CRC}} \\
			\hline
			\textbf{Clase} & \textbf{Conversación} \\
			\hline
			\parbox[t]{0.45\linewidth}{\textbf{Responsabilidades:} \\ 
				Iniciar una nueva sesión de chat \\ 
				Almacenar consultas y respuestas \\ 
				Permitir continuación de una conversación existente \\ 
				Eliminar una conversación del historial} 
			& 
			\parbox[t]{0.45\linewidth}{\textbf{Colaboración:} \\ 
				Usuario \\ 
				Historial \\ 
				Backend \\ 
				BaseDeDatos} \\
			\hline
		\end{longtable}
		
		\begin{longtable}{|l|l|}
			\caption{Tarjeta CRC: Historial} \label{tablacrc9} \\
			\hline
			\multicolumn{2}{|c|}{\textbf{Tarjeta CRC}} \\
			\hline
			\textbf{Clase} & \textbf{Historial} \\
			\hline
			\parbox[t]{0.45\linewidth}{\textbf{Responsabilidades:} \\ 
				Listar todas las conversaciones de un usuario \\ 
				Proporcionar metadatos de conversaciones (título, fecha) \\ 
				Permitir selección de una conversación específica} 
			& 
			\parbox[t]{0.45\linewidth}{\textbf{Colaboración:} \\ 
				Usuario \\ 
				Conversación \\ 
				Backend \\ 
				BaseDeDatos} \\
			\hline
		\end{longtable}
		
		\begin{longtable}{|l|l|}
			\caption{Tarjeta CRC: Chat} \label{tablacrc10} \\
			\hline
			\multicolumn{2}{|c|}{\textbf{Tarjeta CRC}} \\
			\hline
			\textbf{Clase} & \textbf{Chat} \\
			\hline
			\parbox[t]{0.45\linewidth}{\textbf{Responsabilidades:} \\ 
				Mostrar la interfaz de chat activo \\ 
				Permitir ingreso de consultas en lenguaje natural \\ 
				Visualizar consultas y respuestas (texto e imágenes) \\ 
				Indicar estado de procesamiento} 
			& 
			\parbox[t]{0.45\linewidth}{\textbf{Colaboración:} \\ 
				Conversación \\ 
				Backend \\ 
				MicroservicioMAS} \\
			\hline
		\end{longtable}
		
		\begin{longtable}{|l|l|}
			\caption{Tarjeta CRC: Backend} \label{tablacrc11} \\
			\hline
			\multicolumn{2}{|c|}{\textbf{Tarjeta CRC}} \\
			\hline
			\textbf{Clase} & \textbf{Backend} \\
			\hline
			\parbox[t]{0.45\linewidth}{\textbf{Responsabilidades:} \\ 
				Gestionar endpoints REST para autenticación y chat \\ 
				Coordinar comunicación entre frontend y MicroservicioMAS \\ 
				Almacenar y recuperar datos de conversaciones \\ 
				Proteger rutas con autenticación} 
			& 
			\parbox[t]{0.45\linewidth}{\textbf{Colaboración:} \\ 
				Usuario \\ 
				Autenticador \\ 
				Conversación \\ 
				Historial \\ 
				Chat \\ 
				MicroservicioMAS \\ 
				BaseDeDatos} \\
			\hline
		\end{longtable}
		
		\begin{longtable}{|l|l|}
			\caption{Tarjeta CRC: BaseDeDatos} \label{tablacrc12} \\
			\hline
			\multicolumn{2}{|c|}{\textbf{Tarjeta CRC}} \\
			\hline
			\textbf{Clase} & \textbf{BaseDeDatos} \\
			\hline
			\parbox[t]{0.45\linewidth}{\textbf{Responsabilidades:} \\ 
				Almacenar datos de usuarios (credenciales, sesiones) \\ 
				Persistir conversaciones y su historial \\ 
				Proveer acceso a datos del \textit{Diario de la Marina} (vectorial y CSV)} 
			& 
			\parbox[t]{0.45\linewidth}{\textbf{Colaboración:} \\ 
				Usuario \\ 
				Autenticador \\ 
				Conversación \\ 
				Historial \\ 
				Backend \\ 
				Agente Recuperador (FAISS) \\ 
				Agente PandasAi} \\
			\hline
		\end{longtable}
		
		\begin{longtable}{|l|l|}
			\caption{Tarjeta CRC: MicroservicioMAS} \label{tablacrc13} \\
			\hline
			\multicolumn{2}{|c|}{\textbf{Tarjeta CRC}} \\
			\hline
			\textbf{Clase} & \textbf{MicroservicioMAS} \\
			\hline
			\parbox[t]{0.45\linewidth}{\textbf{Responsabilidades:} \\ 
				Recibir consultas del backend \\ 
				Coordinar agentes internos para procesar consultas \\ 
				Devolver respuestas procesadas (texto e imágenes) al backend} 
			& 
			\parbox[t]{0.45\linewidth}{\textbf{Colaboración:} \\ 
				Backend \\ 
				Agente Moderador \\ 
				Agente Recuperador (FAISS) \\ 
				Agente Contextualizador \\ 
				Agente de Validación \\ 
				Agente PandasAi} \\
			\hline
		\end{longtable}
		
		
		
		
\end{addendum}
 % Anexos
	
	% Bibliografía
	\bibliographystyle{IEEEtran}
	\bibliography{library} 
	%\printbibliography
	
\end{document}