\resumen

\parskip 10pt \spacing{1.5} \setlength{\parindent}{0pc}

En la era digital, vastos archivos históricos como el \textit{Diario de la Marina} (1844-1960), crónicas del transporte marítimo cubano, permanecen a menudo como tesoros de información subutilizados debido a la complejidad de su acceso y análisis. Esta investigación se propuso desbloquear este potencial mediante el desarrollo de un innovador sistema multiagente conversacional. El objetivo fue transformar los datos digitalizados, frecuentemente imperfectos por errores de OCR y lenguaje antiguo, en conocimiento interactivo y contextualizado para historiadores y académicos. Empleando la metodología ágil \textit{Extreme Programming} (XP), se construyó una arquitectura de microservicios robusta (React, Django REST \textit{Framework}, \textit{FastAPI}) que orquesta un sistema de IA especializado. Este núcleo inteligente, basado en la arquitectura de Recuperación Aumentada por Generación (RAG) y potenciado por \textit{LangChain}, \textit{embeddings} y bases de datos vectoriales FAISS, permite a los usuarios dialogar en lenguaje natural con la historia, obteniendo no solo respuestas textuales precisas sino también visualizaciones gráficas dinámicas. Las pruebas exhaustivas confirmaron la funcionalidad completa de todos los requisitos y la seguridad. Si bien se cumplieron los objetivos de rendimiento, la latencia inherente a la IA síncrona se identificó como un desafío a optimizar, concluyendo que la sinergia entre sistemas multiagente e IA generativa ofrece una vía prometedora y viable para revitalizar el patrimonio documental.

\textbf{Palabras clave:} Datos Históricos, Inteligencia Artificial Conversacional, Procesamiento de Lenguaje Natural, Recuperación Aumentada por Generación, Sistema Multiagente.