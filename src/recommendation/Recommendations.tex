\suggestions

Como resultado de esta investigación, se identifican varias líneas de acción que permitirían perfeccionar, mantener y extender el sistema multiagente conversacional desarrollado, garantizando su evolución sostenida en el tiempo:

\begin{itemize}
	\item \textbf{Optimización del rendimiento del microservicio de agentes:} Se recomienda la implementación de mecanismos de paralelización o asincronía en la inferencia de modelos de lenguaje, con el fin de reducir la latencia de respuesta y mejorar la experiencia de usuario en consultas complejas.
	
	\item \textbf{Ampliación del corpus documental:} Resulta pertinente integrar otras fuentes históricas digitalizadas (como registros aduanales, prensa regional o documentos de archivo) que enriquezcan el contexto y permitan análisis más amplios sobre el comercio marítimo en el Caribe.
	
	\item \textbf{Incorporación de técnicas de aprendizaje activo:} Se sugiere explorar métodos de retroalimentación explícita por parte de los usuarios y el mismo sitema multiagente para refinar progresivamente la interpretación de consultas y mejorar la relevancia semántica de las respuestas generadas.
	
\end{itemize}

Estas recomendaciones tienen como propósito consolidar el sistema como una herramienta sostenible y extensible, promoviendo su uso tanto en contextos académicos como patrimoniales.

