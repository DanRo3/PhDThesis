\englishabstract

\parskip 10pt \spacing{1.5} \setlength{\parindent}{0pc}

In the digital age, vast historical archives such as the \textit{Diario de la Marina} (1844-1960), chronicles of Cuban maritime transport, often remain underutilized treasures of information due to the complexity of their access and analysis. This research aimed to unlock this potential by developing an innovative conversational multi-agent system. The objective was to transform digitized data, frequently imperfect due to OCR errors and archaic language, into interactive and contextualized knowledge for historians and academics. Employing the Extreme Programming (XP) agile methodology, a robust microservices architecture (React, Django REST Framework, FastAPI) was built to orchestrate a specialized AI system. This intelligent core, based on the Retrieval Augmented Generation (RAG) architecture and powered by LangChain, embeddings, and FAISS vector databases, allows users to dialogue with history in natural language, obtaining not only precise textual answers but also dynamic graphical visualizations. Comprehensive testing confirmed full functionality and security. While performance targets were met, the inherent latency of synchronous AI was identified as a challenge to optimize, concluding that the synergy between multi-agent systems and generative AI offers a promising and viable pathway to revitalize documentary heritage

\textbf{Keywords:} Conversational Artificial Intelligence, Historical Data, Multi-Agent System, Natural Language Processing, Retrieval Augmented Generation