\introduction

\setlength{\parindent}{0pt} % Elimina la sangría
\setlength{\parskip}{10pt}  % Aumenta el espacio entre párrafos

La digitalización masiva de documentos históricos ha transformado el acceso a fuentes de información que, durante siglos, permanecieron confinadas a archivos físicos. Este proceso ha generado una explosión de datos digitales provenientes de colecciones diversas, como periódicos históricos, manuscritos y registros oficiales. 
%Un ejemplo emblemático es el \textit{Diario de la Marina} (Cuba, 1844–1960), un testimonio clave de la historia política y cultural del Caribe~\cite{wikipedia_diario}. 
Sin embargo, la transición del medio impreso al digital no ha estado exenta de desafíos. La conversión de estos materiales ha dado lugar a grandes volúmenes de datos no estructurados, caracterizados por su heterogeneidad y la falta de metadatos estandarizados, lo que dificulta su procesamiento automatizado~\cite{lucidea_metadata}.\\
Este fenómeno genera retos de índole técnica, como la conservación a largo plazo y la compatibilidad entre formatos, así como de naturaleza conceptual, asociados a la obtención y contextualización del conocimiento para propósitos académicos y culturales~\cite{lucidea_metadata}. La incapacidad de estructurar y analizar eficientemente estos datos limita su potencial como recurso para la investigación histórica, el análisis lingüístico y el estudio de patrones sociales.\\
%Frente a esta problemática, la inteligencia artificial emerge como una solución prometedora. En particular, los modelos de lenguaje de gran escala (LLMs) ofrecen nuevas posibilidades para la interpretación y generación de texto con alto grado de sofisticación~\cite{proposed_llm}. Sin embargo, su aplicación aislada no es suficiente para abordar tareas complejas como categorización, enriquecimiento semántico y búsqueda contextual~\cite{proposed_llm}.\\ 
Un ejemplo emblemático sobre esta situación es el \textit{Diario de la Marina}, autodenominado «El decano de la prensa cubana», fue un referente de la prensa en La Habana entre 1844 y 1960. De carácter conservador, documentó eventos políticos y sociales, además de registrar información valiosa sobre actividades económicas, como el transporte marítimo, pilar fundamental de la historia comercial de Cuba~\cite{fotosdlahabana}.\\
La digitalización de sus páginas ha permitido preservar este patrimonio histórico, aunque con limitaciones significativas. Factores como la diversidad de tipografías antiguas, la calidad variable de impresión y el estado de conservación de los documentos han generado errores de transcripción y desorganización de la información~\cite{enhancing_degraded}. En el caso específico del transporte marítimo —rutas, cargamentos, puertos y fechas—, esta falta de estructura dificulta su análisis sistemático y su uso en investigaciones. Por ello, la mera digitalización no basta: es necesario aplicar procesos avanzados de interpretación y contextualización que permitan convertir estos datos en conocimiento útil y accesible para responder a consultas especializadas~\cite{ai_assisted, enhancing_degraded}.

A partir de este panorama, se identifican tres ejes problemáticos interrelacionados:

\begin{itemize}
	\item \textbf{Desestructuración y heterogeneidad de los datos digitalizados:} La conversión de documentos impresos antiguos —como el \textit{Diario de la Marina}— a formatos digitales ha generado grandes volúmenes de información no estructurada, afectados por errores de reconocimiento óptico de caracteres (OCR), tipografías irregulares y la ausencia de metadatos estandarizados. Esta situación dificulta su análisis computacional y limita su reutilización en entornos académicos o científicos.
	
	\item \textbf{Falta de contextualización semántica para propósitos analíticos:} Aunque se dispone de los datos digitalizados, estos carecen de mecanismos que permitan comprender su significado en contexto. Información como rutas marítimas, nombres de embarcaciones o fechas relevantes no está categorizada ni vinculada semánticamente, lo que impide su integración en flujos de análisis históricos o culturales más amplios.
	
	\item \textbf{Limitaciones de las soluciones tradicionales frente a la complejidad del dominio:} Las herramientas clásicas de recuperación de información no logran abordar la riqueza y ambigüedad inherente al lenguaje histórico.
\end{itemize}


\textbf{Problema de la investigación}

A partir de la situación problemática descrita se identifica el siguiente problema de la investigación: ¿Cómo interpretar consultas en lenguaje natural para realizar análisis estadísticos sobre datos del transporte marítimo recogidos en el \textit{Diario de la Marina}?

Con base en lo anterior, se define como \textbf{objeto de estudio} el proceso de análisis estadístico de datos a través de consultas en lenguaje natural , y como \textbf{campo de acción} el análisis estadístico de datos mediante el uso de sistemas multiagente conversacionales y modelos neuronales del lenguaje.

\textbf{Objetivo General de la Investigación}

Desarrollar un sistema multiagente conversacional que permita interpretar y contextualizar automáticamente los datos no estructurados del transporte marítimo recogidos en el Diario de la Marina,para su transformación en conocimiento estructurado que facilite el análisis de patrones comerciales y la comprensión de las dinámicas económicas del Caribe entre 1844 y 1960.

\textbf{Objetivos Específicos}

A partir del planteamiento del objetivo de investigación, se definen los siguientes objetivos específicos:

\begin{itemize}
	\item Analizar los referentes teóricos y metodológicos sobre la transformación de información no estructurada a lenguaje natural, revisando enfoques de inteligencia artificial y sistemas conversacionales aplicados a datos históricos.
	\item Realizar la identificación de requisitos, análisis y diseño del sistema multiagente conversacional, definiendo sus componentes, interacciones y flujos de trabajo.
	\item Desarrollar un sistema multiagente conversacional que contribuya a la transformación y el análisis de la información no estructurada extraída del \textit{Diario de la Marina}, enfocándose en los datos del transporte marítimo.
	\item Validar el correcto funcionamiento del sistema multiagente conversacional, aplicando métricas de evaluación y pruebas de software para garantizar su calidad, precisión y usabilidad.
\end{itemize}

\textbf{Métodos de Investigación}

Para llevar a cabo esta investigación se aplicaron métodos teóricos y empíricos de la investigación científica, los cuales se relacionan con el desarrollo de un sistema multiagente conversacional para la transformación e interacción con datos históricos. A continuación, se detallan:

\textbf{Métodos Teóricos}

\begin{enumerate}
	\item \textbf{Analítico-sintético.} Permitió analizar, sintetizar y evaluar el proceso de transformación de datos no estructurados provenientes del \textit{Diario de la Marina}, desde un enfoque centrado en la aplicación de técnicas de procesamiento de lenguaje natural y sistemas multiagentes. Con este método se identificó la esencia del problema de investigación, analizando los componentes del proceso de digitalización, las limitaciones de las técnicas actuales de reconocimiento óptico de caracteres (OCR) y las necesidades de contextualización de los datos del transporte marítimo para su uso efectivo.
	
	\item \textbf{Hipotético-deductivo.} Se empleó para identificar las variables clave involucradas en la interacción conversacional con datos históricos y sus interrelaciones, especialmente aquellas relacionadas con el diseño de sistemas multiagentes su implementación y el procesamiento de consultas en lenguaje natural. Este método facilitó la formulación de supuestos sobre cómo los agentes colaborativos pueden mejorar la interpretación y contextualización de la información no estructurada.
	
	\item \textbf{Histórico-lógico.} Se aplicó para revisar la evolución de las tecnologías de digitalización de documentos históricos, el desarrollo de sistemas conversacionales basados en inteligencia artificial y el uso de datos del transporte marítimo en contextos históricos. Este enfoque permitió reconocer los avances teórico-prácticos en el área, así como las limitaciones actuales en la gestión de datos no estructurados extraídos de fuentes como el \textit{Diario de la Marina}.
\end{enumerate}

\textbf{Métodos Empíricos}

\begin{enumerate}
	\item \textbf{Análisis documental.} Consistió en la revisión de la literatura relacionada con la transformación de datos no estructurados, el diseño de sistemas multiagentes y las aplicaciones de modelos de lenguaje en la interpretación de textos históricos. Incluyó el estudio de enfoques, algoritmos y herramientas de inteligencia artificial utilizadas en tareas de procesamiento de lenguaje natural y extracción de conocimiento a partir de documentos digitalizados.
	
	\item \textbf{Experimental.} Para validar los resultados del sistema multiagente conversacional desarrollado, se evaluó su capacidad para interpretar consultas en lenguaje natural y contextualizar los datos del transporte marítimo provenientes del \textit{Diario de la Marina}. Los resultados obtenidos se compararon mediante un conjunto de métricas de evaluación, incluyendo precisión, tiempo de respuesta, tasa de éxito en la resolución de tareas y coherencia semántica, todas ellas ajustadas a la calidad de las respuestas generadas y a la satisfacción del usuario.
\end{enumerate}

\textbf{Estructura del Documento}

El documento está organizado en introducción, tres capítulos, conclusiones, recomendaciones, bibliografía y anexos. A continuación, se describe el contenido abordado en cada capítulo:

\begin{itemize}
	\item \textbf{Capítulo 1. Fundamentación Teórica:} Se realiza un estudio y análisis de los diferentes métodos y técnicas para el procesamiento de datos no estructurados provenientes de documentos históricos, con énfasis en la digitalización del \textit{Diario de la Marina}. Se analizan los fundamentos teóricos relacionados con la transformación de datos mediante técnicas de inteligencia artificial, centrándose en el uso de sistemas multiagentes y modelos de lenguaje para interpretar textos históricos. De igual manera, se revisan los principales referentes teóricos sobre el procesamiento de lenguaje natural (PLN), la estructuración de información no estructurada y la interacción conversacional con usuarios. Finalmente, se evalúan los desafíos específicos asociados a los datos del transporte marítimo (rutas, puertos, fechas) extraídos de fuentes digitalizadas, identificando las limitaciones de las técnicas actuales como el Reconocimiento Óptico de Caracteres (OCR).
	
	\item \textbf{Capítulo 2. Análisis, diseño e implementación de la propuesta de solución:} En este capítulo, se desarrolla un sistema multiagente conversacional diseñado para interpretar y contextualizar automáticamente los datos no estructurados del transporte marítimo del \textit{Diario de la Marina}. Se modelan y describen la arquitectura empleada en el sistema multiagente, requisitos funcionales y no funcionales, así como los patrones de diseño implementados en la solución. Finalmente, se presentan las conclusiones parciales del capítulo, destacando las decisiones de diseño y los resultados preliminares de la implementación.
	
	\item \textbf{Capítulo 3. Pruebas de software:} Se desarrolla un conjunto de experimentos utilizando una muestra representativa de datos digitalizados del \textit{Diario de la Marina}, enfocándose en registros del transporte marítimo, y se discuten los principales resultados en cuanto a la eficacia del sistema multiagente al procesar consultas en lenguaje natural. Con la arquitectura óptima seleccionada, se evalúa el sistema propuesto en escenarios prácticos, como la respuesta a preguntas sobre rutas marítimas, puertos y fechas históricas. Primero, se valida el sistema con una muestra controlada de datos extraídos del periódico y se comparan los resultados con enfoques tradicionales del estado del arte (por ejemplo, búsquedas manuales o sistemas no conversacionales).Se realizan pruebas funcionales, unitarias, de seguridad y de rendimiento.
\end{itemize}


	











