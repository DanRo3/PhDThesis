\mathnotation
Para lograr una descripci�n homog�nea de los principales conceptos y definiciones se debe establecer la notaci�n b�sica de modo que se comprendan los referentes te�ricos y las contribuciones de la presente investigaci�n. A continuaci�n se detallan elementos generales de la notaci�n matem�tica a utilizar en este trabajo:

\begin{itemize}
	\item $V$: Vocabulario de entrada.
	\item $N$: Cantidad de ejemplos de entrenamiento.
	\item $d$: Dimensi�n o cantidad de rasgos de un ejemplo de entrenamiento. En una entrada de texto puede representar el n�mero de \textit{tokens} que conforma el documento de entrada.
	\item $d_{model}$: Dimensi�n del vector \textit{embedding} que representa una palabra o \textit{token}.
	\item $\mathbf{x}$: Vector que representa un ejemplo de entrenamiento, tal que $\mathbf{x}\in {\rm I\!R}^{d}$
	\item $X$: Conjunto de entrenamiento, tal que $X=\{\mathbf{x}_1,\mathbf{x}_2,\mathbf{x}_3,..., \mathbf{x}_N\}$.
	\item $W$: Matriz que representa los par�metros de una ANN, tal que $W \in {\rm I\!R}^{A \times K}$, siendo $A$ el n�mero de filas y $K$ el n�mero de columnas.
	\item $W_{i:}$ Representa el vector de la i-�sima fila de la matriz $W$.
	\item $W_{:j}$ Representa el vector de la j-�sima columna de la matriz $W$.
	\item $w_{i,j}$ Elemento de la matriz $W$ que se encuentra en la i-�sima fila y j-�sima columna, tal que $0 \le i < A$ y $0 \le j < K$.
	\item $F(W,X)$: Funci�n objetivo de una ANN a optimizar de forma tal que $F(.):{\rm I\!R}^{A \times K} \rightarrow {\rm I\!R}$.
\end{itemize}